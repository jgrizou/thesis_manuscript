%!TEX root = ../thesis.tex

\begin{vcenterpage}
\noindent\rule[2pt]{\textwidth}{0.5pt}
\begin{center}
{\large\textbf{\thesistitle\\}}
\end{center}
{\large\textbf{Abstract:}} 

This thesis investigates how a robot can be taught a new task from human instructions and without knowing beforehand how to associate the human communicative signals with their meanings.

We present an algorithm allowing a user to teach a robot a new task using unknown instruction signals. For this work, we consider different scenarios where a human teacher uses initially unclassified speech words, whose associated meaning can be a feedback (good/bad) or a guidance (go left, right, up, ...). We present computational results, within an inverse reinforcement learning framework, showing that: a) it is possible to learn the meaning of unknown and noisy teaching signals, as well as a new task at the same time, b) it is possible to reuse the acquired knowledge about teaching signals for learning new tasks and c) even in the case where the robot initially knows some of the teaching signals' meanings, the use of extra unknown teaching signals improves learning efficiency. We further introduce an efficient planning strategy that exploits the task and instruction uncertainty to allow more efficient learning sessions.

The theoretical and empirical work presented in this thesis constitute an important first step towards flexible personalized teaching interfaces, a key for the future of personal robotics.

% We present an algorithm allowing a user to teach a robot a new task using unknown instruction signals. For this work, we consider different scenarios where a human teacher uses initially unclassified speech words, whose associated meaning can be a feedback (good/bad) or a guidance (go left, right, up, ...). We present computational results, within an inverse reinforcement learning framework, showing that: a) it is possible to learn the meaning of unknown and noisy teaching signals, as well as a new task at the same time, b) it is possible to reuse the acquired knowledge about teaching signals for learning new tasks and c) even in the case where the robot initially knows some of the teaching signals' meanings, the use of extra unknown teaching signals improves learning efficiency. We further introduce an efficient planning strategy that exploits the task and instruction uncertainty to allow more efficient learning sessions.

% We introduce a new experimental setup, where two humans have to collaborate to solve a task. The channels of communication they can use are constrained and force them to invent and agree on a shared interaction protocol in order to solve the task. These constraints allow us to analyze how a communication protocol is progressively established through the interplay and history of individual actions.

% {\large\textbf{Keywords:}} Keyword 1, Keyword2.\\

\noindent\rule[2pt]{\textwidth}{0.5pt}
\end{vcenterpage}