%!TEX root = ../../thesis.tex
\renewcommand{\chapterpath}{\allchapterspath/limitations}
\renewcommand{\imgpath}{\chapterpath/img}

\chapter{Limitations and Extensions}
\label{chapter:limitations}
\minitoc

Context - Why now: We have shown an algortihm that seems to work

Need - Why the reader: we need to know what are the hidden assumption that are really done and how to overcome them. And potential way to go into more complex problems

Task - Why me: I investigated this question and ran some proof of concept experiment that demonstrate my points.

Object - Why this chapter: We list a number of limitation (continuous state, finite set of task, pre-defined unique interaction frame, human is not affected by agent behavior). We then provide ideas to solve that problem and illustrate with some more or less toy example.

Findings - What: We found that there is some limitation and that some direction are shown to be working in simple scenarios.

Conclusions - So what: Let's go work on finding the limitation

Perspectives - What now: 

%%%%%%%%%%%%%%%%%%%%%%%%%%%%%%%%%%%%%%%%%%%%%%
%%%%%%%%%%%%%%%%%%%%%%%%%%%%%%%%%%%%%%%%%%%%%%
%%%%%%%%%%%%%%%%%%%%%%%%%%%%%%%%%%%%%%%%%%%%%%
%%%%%%%%%%%%%%%%%%%%%%%%%%%%%%%%%%%%%%%%%%%%%%
%%%%%%%%%%%%%%%%%%%%%%%%%%%%%%%%%%%%%%%%%%%%%%
\section{Continuous state space}
the frontiers xp

%%%%%%%%%%%%%%%%%%%%%%%%%%%%%%%%%%%%%%%%%%%%%%
%%%%%%%%%%%%%%%%%%%%%%%%%%%%%%%%%%%%%%%%%%%%%%
%%%%%%%%%%%%%%%%%%%%%%%%%%%%%%%%%%%%%%%%%%%%%%
%%%%%%%%%%%%%%%%%%%%%%%%%%%%%%%%%%%%%%%%%%%%%%
%%%%%%%%%%%%%%%%%%%%%%%%%%%%%%%%%%%%%%%%%%%%%%
\section{Finite set of Hypothesis}
a simple particle filter

%%%%%%%%%%%%%%%%%%%%%%%%%%%%%%%%%%%%%%%%%%%%%%
%%%%%%%%%%%%%%%%%%%%%%%%%%%%%%%%%%%%%%%%%%%%%%
%%%%%%%%%%%%%%%%%%%%%%%%%%%%%%%%%%%%%%%%%%%%%%
%%%%%%%%%%%%%%%%%%%%%%%%%%%%%%%%%%%%%%%%%%%%%%
%%%%%%%%%%%%%%%%%%%%%%%%%%%%%%%%%%%%%%%%%%%%%%
\section{Pre-defined interaction frame}
first freeze the number of task and find the correct interaction frame from a set of interaction frame
second, have two set, one for the frame one for the task

%%%%%%%%%%%%%%%%%%%%%%%%%%%%%%%%%%%%%%%%%%%%%%
%%%%%%%%%%%%%%%%%%%%%%%%%%%%%%%%%%%%%%%%%%%%%%
%%%%%%%%%%%%%%%%%%%%%%%%%%%%%%%%%%%%%%%%%%%%%%
%%%%%%%%%%%%%%%%%%%%%%%%%%%%%%%%%%%%%%%%%%%%%%
%%%%%%%%%%%%%%%%%%%%%%%%%%%%%%%%%%%%%%%%%%%%%%
\section{Human in the loop}
how does the agent behavior affect the algorithm assumption on human behavior

Assumption: The properties of the signals do not change wrt. the behavior of the agent

Users comply with the frame implemented. Same meaning, optimal strategies, timing...

discuss how the setup can be used in HRI, but also semiotic stuff...

%%%%%%%%%%%%%%%%%%%%%%%%%%%%%%%%%%%%%%%%%%%%%%
%%%%%%%%%%%%%%%%%%%%%%%%%%%%%%%%%%%%%%%%%%%%%%
%%%%%%%%%%%%%%%%%%%%%%%%%%%%%%%%%%%%%%%%%%%%%%
%%%%%%%%%%%%%%%%%%%%%%%%%%%%%%%%%%%%%%%%%%%%%%
%%%%%%%%%%%%%%%%%%%%%%%%%%%%%%%%%%%%%%%%%%%%%%
\section{Word properties}
xp with a grid world versus a real maze with many wall and see that random becomes a joke




