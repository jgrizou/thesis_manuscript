%!TEX root = ../../thesis.tex
\renewcommand{\chapterpath}{\allchapterspath/humanexperiment}
\renewcommand{\imgpath}{\chapterpath/img}

\chapter{Experiment with human subjects}
\label{chapter:humanexperiment}
\minitoc

ICDL paper with Anna-Lisa Vollmer and K.J. Rohlfing.

Context - Why now: We define a new challenge of interaction without pre-coordination for robot but does human can do this? A lot of work has been done in teaching devices from human instruction. However very few experiments considered the option to put a human in place of the robot with the quite restricted perception and/or capabilities, as robot have currently.

Need - Why the reader: We study the effect of how the H-H partners manage to work together.

Task - Why us: We performed an experiment where two user have asymmetric role and restricted communication

Object- Why this chapter: The following chapters presents in more detailed the related work concerning human experiments, the settings and interaction protocol.

Findings - What: We show that people manage to successfully interact and find out that they usually try to frame the interaction into universally known interaction patterns. We introduce the terms of interaction frame here. And find out that some frame are more recurrent and easily understood than the others. They solve the problem by hypothesizing on what the user would want to say in a given situation and match with the history of event for each button. Hypothesis generation and testing.

Conclusions - So what: Based on such observations, we could imagine a robot that is equipped with one or several interaction frame and try to find out what signal means what based on the history of interaction.

Perspectives - What now: We should now study if this approach could work in practice.

%%%%%%%%%%%%%%%%%%%%%%%%%%%%%%%%%%%%%%%%%%%%%%
%%%%%%%%%%%%%%%%%%%%%%%%%%%%%%%%%%%%%%%%%%%%%%
%%%%%%%%%%%%%%%%%%%%%%%%%%%%%%%%%%%%%%%%%%%%%%
%%%%%%%%%%%%%%%%%%%%%%%%%%%%%%%%%%%%%%%%%%%%%%
%%%%%%%%%%%%%%%%%%%%%%%%%%%%%%%%%%%%%%%%%%%%%%
\section{Related work}

\subsection{Experimental semiotic}

%%%%%%%%%%%%%%%%%%%%%%%%%%%%%%%%%%%%%%%%%%%%%%
%%%%%%%%%%%%%%%%%%%%%%%%%%%%%%%%%%%%%%%%%%%%%%
%%%%%%%%%%%%%%%%%%%%%%%%%%%%%%%%%%%%%%%%%%%%%%
%%%%%%%%%%%%%%%%%%%%%%%%%%%%%%%%%%%%%%%%%%%%%%
%%%%%%%%%%%%%%%%%%%%%%%%%%%%%%%%%%%%%%%%%%%%%%
\section{The Collaborative Construction Game}

%%%%%%%%%%%%%%%%%%%%%%%%%%%%%%%%%%%%%%%%%%%%%%
%%%%%%%%%%%%%%%%%%%%%%%%%%%%%%%%%%%%%%%%%%%%%%
%%%%%%%%%%%%%%%%%%%%%%%%%%%%%%%%%%%%%%%%%%%%%%
%%%%%%%%%%%%%%%%%%%%%%%%%%%%%%%%%%%%%%%%%%%%%%
%%%%%%%%%%%%%%%%%%%%%%%%%%%%%%%%%%%%%%%%%%%%%%
\section{Experiments}

%%%%%%%%%%%%%%%%%%%%%%%%%%%%%%%%%%%%%%%%%%%%%%
%%%%%%%%%%%%%%%%%%%%%%%%%%%%%%%%%%%%%%%%%%%%%%
%%%%%%%%%%%%%%%%%%%%%%%%%%%%%%%%%%%%%%%%%%%%%%
%%%%%%%%%%%%%%%%%%%%%%%%%%%%%%%%%%%%%%%%%%%%%%
%%%%%%%%%%%%%%%%%%%%%%%%%%%%%%%%%%%%%%%%%%%%%%
\section{Lessons Learned}

Introduce interaction frame here
What do we learn from this? 
How can it be used to build an algorithm?



