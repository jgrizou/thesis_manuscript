%!TEX root = ../../thesis.tex
\define{\chapterpath}{\allchapterspath/conclusion}
\define{\imgpath}{\chapterpath/img}

\chapter{Conclusion}
\label{chapter:conclusion}
\minitoc


Context - Why now: we have developed a complete solution

Need - Why the reader: we must wrap up the all shit 

Task - Why me: We have show that there is some problem that were not or very few people tackled it before and tried to identify the problem, how to solve, implemented it and shown that it works, also with BCI, and presented many limitation and some direction to overcome them

Object- Why this chapter: Same thing as above

Findings - What: we have a complete document that present a complete theory

Conclusions - So what: well you can try to use this shit and make real proof of it, maybe study it in different settings, and different signals and more symbolic stuff which are easier and you will say it is brand new, and publish more conference paper that will say again the same thing as we did in a different way showing that you changed the world.

Perspectives - What now: let's go further with having agent learning meanings

The transition from raw input to higher level meaning is however not much investigated in science. Here our agent or robot already have the ability to categorize state, make plans and select what is relevant from the environment.

The only example known today is the Siri interface from apple that learn to adapt to each particular user voice specificities.



%%%%%%%%%%%%%%%%%%%%%%%%%%%%%%%%%%%%%%%%%%%%%%
%%%%%%%%%%%%%%%%%%%%%%%%%%%%%%%%%%%%%%%%%%%%%%
%%%%%%%%%%%%%%%%%%%%%%%%%%%%%%%%%%%%%%%%%%%%%%
%%%%%%%%%%%%%%%%%%%%%%%%%%%%%%%%%%%%%%%%%%%%%%
%%%%%%%%%%%%%%%%%%%%%%%%%%%%%%%%%%%%%%%%%%%%%%

We are not doing perfect science, all this is really empirical and lacks of clean mathetical work, such as proof and studies of the properties of the system. 

In this paper we have shown that, given a limited number of possible tasks, it is possible to solve sequential tasks using human feedback without defining a map between feedback signals and their meaning beforehand. The proposed algorithm optimizes a pseudo-likelihood function and performs active planing according to the the uncertainty in the task and meaning spaces. Indeed, taking into account this uncertainty is crucial to solve the task efficiently and to recover the actual meanings. This combination allows: 
\begin{inparaenum}[a)]
\item a human to start interacting with a system without calibration;
\item to automatically adapt calibration time to the user needs which can even outperform fixed calibration procedures; 
\item to adapt to the uncertainty of the information source from scratch.
\end{inparaenum}
We showed the applicability of the approach to brain-machine interfaces based on error potentials which could work out of the box without calibration, a long-desired property of this type of systems. 

% look around you, there is computers everywhere most of them are connect to internet and can access globaly to an infinte amount of information. How do you get your phone to display the infromation that you need? you need to open an ap and enter the that describe best your need. For the most skilled person they can code their own application that display one bits of information they need. But think of the time it take and the many skills needed to have this phone makes what you wants it to do. That is huge, if this phone where really smartphones you would just had to ask right? they have all the hardware required to display things on their screen, play any song you want and go over the internet looking for any bits of information. Apple and Google, two worlwide multinational compagny started t devellop system that learns how to react to speech request formt their user, whether to set an alarm clock or to look for a restaurant nearby. Does it work? surpringly well for the common business you do with you smarphone such as setting an alarm clock, but that well for more complex task and they just can't learn anything new, they can't learn to draw freely on the screen. nul 

% car are braking alone, internet works thank to many iclever algortihm, some of them adapt themselves to the incoming traffic and it is often impossible to predict their outcome 