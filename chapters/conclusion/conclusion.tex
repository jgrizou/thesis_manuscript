%!TEX root = ../../thesis.tex
\renewcommand{\chapterpath}{\allchapterspath/conclusion}
\renewcommand{\imgpath}{\chapterpath/img}

\chapter{Conclusion}
\label{chapter:conclusion}
\minitoc


Context - Why now: we have developed a complete solution

Need - Why the reader: we must wrap up the all shit 

Task - Why me: We have show that there is some problem that were not or very few people tackled it before and tried to identify the problem, how to solve, implemented it and shown that it works, also with BCI, and presented many limitation and some direction to overcome them

Object- Why this chapter: Same thing as above

Findings - What: we have a complete document that present a complete theory

Conclusions - So what: well you can try to use this shit and make real proof of it, maybe study it in different settings, and different signals and more symbolic stuff which are easier and you will say it is brand new, and publish more conference paper that will say again the same thing as we did in a different way showing that you changed the world.

Perspectives - What now: let's go further with having agent learning meanings

The transition from raw input to higher level meaning is however not much investigated in science. Here our agent or robot already have the ability to categorize state, make plans and select what is relevant from the environment.

The only example known today is the Siri interface from apple that learn to adapt to each particular user voice specificities.

%%%%%%%%%%%%%%%%%%%%%%%%%%%%%%%%%%%%%%%%%%%%%%
%%%%%%%%%%%%%%%%%%%%%%%%%%%%%%%%%%%%%%%%%%%%%%
%%%%%%%%%%%%%%%%%%%%%%%%%%%%%%%%%%%%%%%%%%%%%%
%%%%%%%%%%%%%%%%%%%%%%%%%%%%%%%%%%%%%%%%%%%%%%
%%%%%%%%%%%%%%%%%%%%%%%%%%%%%%%%%%%%%%%%%%%%%%
