%!TEX root = ../../thesis.tex
\define{\chapterpath}{\allchapterspath/bci}
\define{\imgpath}{\chapterpath/img}

\chapter{Application to Brain Computer Interaction}
\label{chapter:bci}
\minitoc

With I\~naki Iturrate and Luis Montesano.

Context - Why now: We have a working algorithm, in simulation or with pretty god data

Need - Why the reader: we want to test it on a realistic setup with signal which make sense to use in tis context, BCI seems a promising direction

Task - Why us: Together with I\~naki Iturrate and Luis Montesano we adapted the algorithm to work with EEG signal on a grid world setup.

Object- Why this chapter:  We present first the EEG paradigm and related work in the domain. Then show simulated experiment using prerecorded data, and finally show some run with some real users.

Findings - What: It works online, with naive subject, and without pre-calibration, but the performance are no the same than in simulation

Conclusions - So what: We can hope to release this algorithm in the real word and make it useful to use. EEG are among the hardest signal to classify, if we can do it with them, we sure can do it with other king of signals.

Perspectives - What now: That are only toy problem, discrete state, discrete action, where should we go next? What are the current limitation?

Recent works have explored the use of brain signals to directly control virtual and robotic agents in sequential tasks. So far in such brain-computer interfaces (BCIs), an explicit calibration phase was required to build a decoder that translates raw EEG signals from the brain of each user into meaningful feedback signals.
%
This paper proposes a method that removes the need for such a calibration phase, and allows a user to control an agent to solve a sequential task.
%
The proposed method assumes a distribution of possible tasks, and infers the interpretation of EEG signals and the task by selecting the hypothesis which best explains the history of interaction. 
%
Also, we use a measure of uncertainty of the task and of the EEG signal interpretation as an exploratory reward for a planning strategy, and this speeds up learning by guiding the system to regions that help disambiguate among task hypotheses.
%
We report experiments where four users control, by means of a BCI, an agent on a virtual world to reach a target without any previous calibration process.

%%%%%%%%%%%%%%%%%%%%%%%%%%%%%%%%%%%%%%%%%%%%%%
%%%%%%%%%%%%%%%%%%%%%%%%%%%%%%%%%%%%%%%%%%%%%%
%%%%%%%%%%%%%%%%%%%%%%%%%%%%%%%%%%%%%%%%%%%%%%
%%%%%%%%%%%%%%%%%%%%%%%%%%%%%%%%%%%%%%%%%%%%%%
%%%%%%%%%%%%%%%%%%%%%%%%%%%%%%%%%%%%%%%%%%%%%%
\section{BCI control using Error Related Potential}

evaluation scenarios were tested with two different types of signals: artificial datasets, and real ErrP datasets recorded from previous experiments \cite{iturrate2013task}.

\paragraph{EEG datasets}
Once the algorithm was evaluated with artificial datasets, we tested the feasibility of the proposed self-calibration approach using real ErrP datasets. The objective of this analysis is to study the scalability of our method to EEG data, which may have different properties than our artificial dataset. 

The EEG data were recorded in a previous study \cite{iturrate2013task} where participants monitored on a screen the execution of a task where a virtual device had to reach a given goal. The motion of the device could be correct (towards the goal) or erroneous (away from the goal). The subjects were asked to mentally assess the device movements as erroneous or non-erroneous. The EEG signals were recorded with a gTec system with 32 electrodes distributed according to an extended 10/20 international system with the ground on FPz and the reference on the left earlobe. The ErrP features were extracted from two fronto-central channels (FCz and Cz) within a time window of $[200,700]$ ms (being 0 ms the action onset of the agent) and downsampled to $32$ Hz. This leaded to a vector of $34$ features.

\paragraph{Comparison with calibration methods}
In order to show the benefit of learning without explicit calibration, we compare our method with the standard supervised BCI calibration procedure. In this calibration procedure, which can last for up to 40 minutes, the experimenter needs to record enough data from the user from several offline runs, where the user is not controlling the agent but just passively assessing its actions.
%
Following the literature on ErrPs \cite{chavarriaga2010learning,iturrate2013task} our training data will consist of 80 percent of positive examples (associated to a correct feedback) and 20 percent of negative examples (associated to an incorrect feedback). Our proposed algorithm is compared with different (but standard) sizes of calibration datasets: 200, 300 and 400 examples.

