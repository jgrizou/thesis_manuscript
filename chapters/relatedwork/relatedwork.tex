%!TEX root = ../../thesis.tex
\define{\chapterpath}{\allchapterspath/relatedwork}
\define{\imgpath}{\chapterpath/img}

\chapter{Related Work}
\label{chapter:relatedwork}
\minitoc

In most robot social learning experiments today, there is a strong decoupling between the process of extracting useful informations from the interaction and the process of learning a new skill from these informations. For example, the human demonstrations are provided in a batch perspective where data acquisition is done before the learning phase. The properties of teaching interactions with a human in the loop are not yet considered in depth.

In this chapter we highlight the difference between systems learning from well-controlled interactions and systems trying to close the interaction loop allowing more flexibility in the interaction process. These issues have began to be addressed in a subfield called \emph{interactive learning}  which combine ideas of social learning with extrinsic and intrinsic motivated learning. With this approach, the robot acquires more autonomy with respect to how to deal with the human in the loop. 

After presenting the related work in interactive learning, we broaden the scope of this work by linking with the computational modeling of language, some aspects of unsupervised learning, and specific works on ad-hoc team whose stated challenge is to enable cooperation without prior-coordination in multi-agent scenarios. Finally, we present related works from the brain computer interfaces (BCI) community.

%%%%%%%%%%%%%%%%%%%%%%%%%%%%%%%%%%%%%%%%%%%%%%
%%%%%%%%%%%%%%%%%%%%%%%%%%%%%%%%%%%%%%%%%%%%%%
%%%%%%%%%%%%%%%%%%%%%%%%%%%%%%%%%%%%%%%%%%%%%%
%%%%%%%%%%%%%%%%%%%%%%%%%%%%%%%%%%%%%%%%%%%%%%
%%%%%%%%%%%%%%%%%%%%%%%%%%%%%%%%%%%%%%%%%%%%%%
\section{Interactive Learning}

In this section, we present a number of works considering the human component into the learning loop. We call this area of research \emph{interactive learning} \cite{nicolescu2003natural,breazeal2004tutelage}. It aims at developing machines that can learn by practical interaction with the user.

% Most of the systems presented in the previous section have not considered in depth the properties of teaching interactions with a human in the loop. The demonstrations are provided in a batch perspective where data acquisition is done before the learning phase.

\emph{Interactive learning} combines ideas of social learning with extrinsic and intrinsic motivated learning. It differs from the works presented in the introduction as both the human and the robot are simultaneously involved in the learning process \cite{kaplan2002robotic,nicolescu2003natural,breazeal2004tutelage,thomaz2008teachable}. Under this approach, the teacher interacts with the robot and provides extra feedback or guidance. In addition, the robot can act to improve its learning efficiency or elicit specific responses from the teacher. Recent developments have considered: extra reinforcement signals \cite{thomaz2008teachable}, action requests \cite{macl09airl}, disambiguation among actions \cite{chernova09jair}, preferences among states \cite{Mason2011}, iterations between practice and user feedback sessions \cite{judah2010reinforcement} and choosing actions that maximize the user feedback \cite{knox2009interactively}.

We decided to split this related work in four categories. Firstly, we present works combining multiple sources of information, such as combining demonstration and feedback. Secondly, we present some studies about the behavior of human when teaching robots. Thirdly, we present works that try to model some aspects of the user behavior or of the protocol. Fourthly, we present works considering an active robot, which try to learn faster from or about the interaction. Finally, we discuss and situate our work in this scope.

% Interactive learning \cite{nicolescu2003natural,breazeal2004tutelage} aims at developing systems that can learn by practical interaction with the user and finds applications in a wide range of fields such as human-robot interaction, tutoring systems or human-machine interfaces.
% This type of learning combines ideas of learning from demonstration \cite{argall09survey}, learning by exploration \cite{thrun1992efficient} and tutor feedback \cite{kaplan2002robotic}. Under this approach the human teacher interacts with the machine and provides extra feedback or guidance. 
% In addition, the device can act to improve its learning efficiency. Approaches have considered: extra reinforcement signals \cite{thomaz2008teachable}, action requests \cite{lopes2009active}, disambiguation among actions \cite{chernova09jair}, preferences among states \cite{Mason2011}, iterations between practice and user feedback sessions \cite{judah2010reinforcement}, and choosing actions that maximize the user feedback \cite{knox2009interactively}. 

\subsection{Combining multiple learning sources}

Researchers have considered mixing different learning paradigms in order to improve the quality of the interaction and of the learning process. They considered:

\begin{itemize}

\item Mixing environmental rewards with human rewards \cite{knox2010combining,griffith2013policy,grave2013learning}. The main problem is to balance the influence of the environmental reward with the human generated reward.

\item Iterations between practice and user feedback sessions \cite{judah2010reinforcement}. The learner first practices the task a few times to learn from environmental reward. Then a user can observe its practice session and classify the policies or actions as good or bad. The learner updates its policy given the reward from the environment and the user critiques, and the process repeats again.

\item Giving some demonstrations first, and having the robot practicing the skill under online human supervision (feedback or guidance) \cite{nicolescu2003natural,pardowitz2007incremental}.

\item Mixing concrete instructions and rewards to balance human efforts with communication efficiency \cite{pilarski2012between}.
% between instruction and reward to balance human effort with communication efficiency. Assistive technology, it is rare that you can control all aspect of the technology at once, for example, many active hand prosthesis have several modes of operation, two finger grip or full hand grip for example, where then fine control should be apply within that mode. Learning the preference of the user in terms of switching time based the past experience on the interaction. If what the system is not good, then the user can manually change it back. 

\item Combining learning from demonstration and mixed initiative control \cite{grollman2007dogged}. Mixed initiative control is when the control can transition smoothly from the demonstrator control to the robot control. In \cite{grollman2007dogged} the authors used this method to teach different behaviors to a robot, such as mirroring the head position with the tail position or to seek for a red ball, using the same algorithm.

\item Combining transfer learning, learning from demonstration and reinforcement learning \cite{taylor2011integrating}.

\item Demonstrating only parts of trajectories. In \cite{akgun12hri}, the users only demonstrate some keyframe positions along the trajectory. The robot can then autonomously infer a trajectory that match with each keyframe position.

\end{itemize}

But researchers also created new learning paradigms, such as learning from users' preferences \cite{Mason2011,akrour2011preference}. In this new paradigm, the system learns the preferences of the human and will pro-actively generalize and apply them autonomously.

In \cite{Mason2011}, the user starts by teleoperating the robot and can mark some states as good or bad. From this data, the robot can create a user profile. Next, the robot can select its own goal without the need for human teleoperation. Once a desirable state of the world has been reached, the human as a possibility to classify the state as good or bad again. The robot can update its user profile, and the process iterates.

% We have developed a robotic system that interacts with the user, and through repeated interactions, adapts to the user so that the system becomes semiautonomous and acts proactively. In this work we show how to design a system to meet a user’s preferences, show how robot pro-activity can be learned and provide an integrated system using verbal instructions. All these behaviors are implemented in a real platform that achieves all these behaviors and is evaluated in terms of user acceptability and efficiency of interaction

In \cite{akrour2011preference,akrour2012april,akrour2014programming,wilson2012bayesian}, the robot demonstrates some candidate policies and ask the human to rank them by preferences. Based on this ranking the algorithm learns a policy scoring function, which is later used to generate new policies. The user ranks these new policies again, and the process iterates. This method differs from the learning from human reinforcement paradigm as the user evaluates full demonstrations. It differs from inverse reinforcement learning because the robot is it-self generating the demonstrations. But more importantly, demonstrations are ranked between them, which differs from the usual assumptions that all demonstrations given to the learning algorithm are equally correct but noisy.


% \cite{akrour2014programming} programming by feedback. The agent it doing the full demonstration and the human ranks the demonstration. With an active selection of the demonstration by the agent. Approximating user utility function, taking noise and errors into account. Looks like IRL but I don't understand how it differ. \cite{akrour2011preference} Preference-based Policy Learning, iterates a four-step process: the robot demonstrates a candidate policy; the expert ranks this policy comparatively to other ones according to her preferences; these preferences are used to learn a policy return estimate; the robot uses the policy return estimate to build new candidate policies. \cite{akrour2012april} Iteratively, the agent selects a new candidate policy and demonstrates it; the expert ranks the new demonstration comparatively to the previous best one; the expert's ranking feedback enables the agent to refine the approximate policy return, and the process is iterated. the lesson learned from the experimental validation of April is that a very limited external information might be sufficient to enable reinforcement learning: while mainstream RL requires a numerical reward to be associated to each state, while inverse reinforcement learning [1,18] requires the expert to demonstrate a sufficiently good policy, April requires a couple dozen bits of information (this trajectory improves / does not improve on the former best one) to reach state of the art results. In most problem some delays is always present between an action and its effective reward. How it compares with eligibility traces \cite{sutton1998reinforcement} in RL? which can significantly speed learning. to handle delayed reward. each time a state is visited, it initiates a short -term memory process, a trace, with then decays gradually over time. This trace marks the state as eligible for learning. Compared to Knox, the reward is at the end not during the exp, make it more difficult to identify which part is correct. \cite{wilson2012bayesian} similar but with active learning as well. use small bits of demonstration. Active request. We consider the problem of learning control policies via trajectory preference queries to an expert. In particular, the learning agent can present an expert with short runs of a pair of policies originating from the same state and the expert then indicates the preferred trajectory. The agent's goal is to elicit a latent target policy from the expert with as few queries as possible. To tackle this problem we propose a novel Bayesian model of the querying process and introduce two methods that exploit this model to actively select expert queries. Experimental results on four benchmark problems indicate that our model can effectively learn policies from trajectory preference queries and that active query selection can be substantially more efficient than random selection.

Most of the methods above consider the users are somehow optimal or at least predictable in their teaching behaviors. However this is not always the case, in next subsection we review studies about the behaviors of humans when teaching robots.

\subsection{How people teach robots}
\label{chapter:related:humanintheloop}

An important challenge is to deal with non-expert humans whose teaching styles can vary considerably. Users may have various expectations and preferences when interacting with a robot and predefined protocols or instructions may bother the user and dramatically decrease the performance of the learning system \cite{thomaz2008teachable,kaochar2011towards,knox2012humans,rouanet2013impact}. These studies show that even when using well-defined protocols, it is important to consider how different instructions can be used for learning.

People will not always respect predefined conventions. Several studies discuss the different behaviors naive teachers use when instructing robots \cite{thomaz2008teachable,Cakmak2010optimality}. When learning from human reinforcement, an important aspect is that the feedback is frequently ambiguous and deviates from the mathematical interpretation of a reward or a sample from a policy. For instance, in the work of A. L. Thomaz et al. \cite{thomaz2008teachable} the teachers frequently gave a positive reward for exploratory actions even if the signal was used by the learner as a standard reward. Also, even if we can define an optimal teaching sequence, humans do not necessarily behave according to those strategies \cite{Cakmak2010optimality}. This is often because the user and the robot do not share the same representation of the problem.

For the specific case of learning from human reinforcement, several works studied how people actually teach by explicit reward and punishment. In \cite{thomaz2006reinforcement}, the authors found that people gave more positive than negative rewards. Also, users tend to use feedback signals to provide guidance to the agent and to encourage the agent in its exploratory actions. In \cite{knox2009design}, the authors show that humans reinforce almost always state-action pairs and not state only. People perceive intentionality in the robot's actions, and therefore human trainers reinforce given the expected long-term returns of an action, i.e. they do not provide a solely immediate reward as reinforcement learning algorithms rely on. Human teachers reinforce what the robot is about to do (they perceive intentionality) or what the robot just did. Therefore the question of how to divide human feedback between future and past actions is not obvious. In addition, human reinforcement behavior is a moving target and cannot be considered as sampled from an immutable hidden reward function. Finally, in \cite{loftinlearning}, the authors studied the role of non explicit feedback. Some users do not always give explicit feedback in response to a robot's action. For example, they have shown that some users are more likely to provide positive feedback than negative feedback. Surprisingly, some users might never give positive feedback. This variety of user profiles makes it difficult to create a general algorithm for learning from human reinforcement. However, if the users are consistent in their strategies, it might be possible to model and exploit them individually. 

Given these observations, considering people as optimal teaching agents seems flawed. Every user may not experience what is optimal for a robot, in a mathematical sense, as optimal. And more importantly each user might experience it differently. There are a number of design principles that have been derived from such experiments to create better interactive learning systems.

\paragraph{Transparency} It is for example important for the user to understand the way the robot ``thinks'' and what are its ``intentions''. A learner displaying its current ``state of mind'' is called a transparent learner \cite{thomaz2008teachable}. A simple example would be a robot that displays its current level of understanding of the task using a colored LED. The robot could also directly vocalize its understanding of some part of the problem, or if it does not understand some words from the teacher \cite{chao2010transparent}. An other option for the robot is to demonstrate what it understands so far while asking for confirmation or correction to the user \cite{cakmak2012designing}.

Also it may be useful to characterize the preferences of users in terms of teaching behavior. In \cite{cakmak2012designing}, Cakmak et al. used human-human experiments to find out which types of question were most often used. Based on their observations, queries about features of the problem were identified as the most common questions. They were also perceived as the smartest when used by the robot. Using this method the robot explicitly tests precise aspects of the task and asks to the teacher: ``can I do that?''.

\paragraph{Controlling the leader/follower balance} Asking feedback from the user is more useful when it allows to differentiate ambiguous states. In \cite{chao2010transparent} active learning is shown to improve the accuracy and efficiency of the teaching process. However active learning may illicit undesirable effects of acceptability by affecting the leader/follower balance during the interaction. In \cite{chao2010transparent}, some people felt uncomfortable when the robot asked too many questions and did not feel like their were the teacher, i.e. the one leading the interaction. As a conclusion, the interaction is best accepted when a proper balance is achieved between autonomy, feedback request and human control. A robot asking a question every step is boring for the user, and asking too infrequently is unpredictable. Finally, allowing users to send feedback to the robot whenever they wanted was preferred by the users but was less efficient for the learning process.

% derived principle: \cite{thomaz2008teachable}  transparency, balance of control leader follower \cite{cakmak2010designing} led to conclusion about balance of autonomy and control. a question every step is boring, and asking sometime is unpredictable. Letting the user send feedback when he wanted was preferred but less efficient.

\paragraph{Testing the robot} As a kind of transparency, it is important for the teacher to be able to ask the learning agent to perform the taught skill to verify and correct it. It allows the user to understand how the agent learns and generalizes from examples. For instance, in \cite{kaochar2011towards} when the participants had the opportunity to test the agent's comprehension, more than half of them preferred testing the student systematically after a new concept or procedure was introduced. They also showed that people tend to test the agents more during the last third of the teaching process.

\transition

To summarize, all teachers are different and most of the time they are not optimal. Even if there are a number of design principles allowing reducing the variability of human teaching behaviors, it is almost impossible to design an experiment where human teaching behavior can be fully predictable. Therefore modeling the users seems a natural next step.

\subsection{User modeling, ambiguous protocols or signals}

% We only focus our attention on the modeling of human users by a robot and during a teaching interaction. 

Modeling the user during the interaction is primordial to adapt to an a priori unknown human. Some works investigate how to learn the user's teaching behavior online \cite{knox2009interactively}, how to learn the meaning of new human signals starting from a set of known signals \cite{macl11simul,loftinlearning}, or how to directly learn the meaning of unknown signals but when the agent has access to a direct measure of its performance \cite{branavan2011learning,kim2012unsupervised,doshi2008spoken}.

In \cite{knox2009interactively}, an artificial agent learns from human reinforcement but the human signals are not treated as a reward in a reinforcement learning problem. Instead the agent models the trainer reinforcement function, and considers it as a moving target. The idea is that the human reinforcement already includes the long-term consequences of the agent's actions, whereas in reinforcement learning the reward act just locally. Therefore, by modeling the user reinforcement function, the agent can act greedily on this function to achieve the desired task. Their approach has been extended to continuous states and actions \cite{vien2013learning}.

In \cite{macl11simul}, the learning agent receives signals of both known and unknown meanings. The agent learns a task using the known information and is then able to infer the associated meaning of the a priori unknown signals. Similarly in \cite{loftinlearning} the agent learns the meaning of non-explicit signals, e.g. when the user do not press any button, but knowing the meaning of all explicit signals. Our problem differs because we do not have access to a subset of signals of known meaning beforehand.

In \cite{branavan2011learning}, the learning agent automatically extract information from a text manual to improve its performance on a task. The agent learns how to play the strategy game Civilization II and it has access to a direct measure of its performance. But the agent also has access to the game manual, which gives some explanation about the game strategy. However the agent does not know how to read and interpret this manual beforehand. The agent then autonomously learns to analyse the text in the manual and to use the information contained in the manual to improve its strategy. In other words, the agent learns the ``language'' of the game manual. While the agent could learn to play the game alone, their results show that \textit{``a linguistically-informed game-playing agent significantly outperforms its language-unaware counterpart''}. Our problem differs because our agent does not have access to a measure of its performance on the task, and can only rely on the unlabeled signals received from the teacher. However we will process much simpler signals without syntactic structure.

Some other works have focused on learning semantic parsers, either from natural language as text \cite{branavan2011learning,kim2012unsupervised} or real speech \cite{doshi2008spoken}. Semantic parsers allow for a more natural human-robot interaction where more advanced set of instructions can be used. In \cite{kim2012unsupervised} the algorithm can produce, with some limitation, previously unseen meaning representation. However these works assume the agent has access to a known and constrained source of information about the task. Either a direct access to its performances \cite{branavan2011learning}, to a reward from a teacher \cite{doshi2008spoken}, or to a tuple (text instruction sentence, state, action sequence) where the instruction describes at a higher level the observed action sequence \cite{kim2012unsupervised}.

\transition

Modeling parts of the user behavior allows an interactive learning agent to adapt to a variety of teaching behaviors. The work presented in this thesis follows along the same lines. We learn mapping between the user's teaching signals and their meanings. But contrary to the works presented above, we simultaneously estimates the desired task, and do not have access to a measure of our performance on the task or to other known sources of information. It allows a user to teach a machine a new task using signals unspecified in advance. As a consequence, if speech is the modality of interaction, our system should handle different languages or even interjections or hand clapping.

\subsection{Active learners and teachers}

Finally another crucial aspect for an efficient interaction is to have both a learner and a teacher seeking to maximize the learning of the learner. We usually call these types of agent \emph{active learners} and \emph{active teachers}. An active learner will seek for situation in which it feels uncertain about what to do, and ask the teacher for more information about that situation. An active teacher will try to provide the most useful demonstrations or instructions to the learning agent. Ideally an active teacher considers the learning capabilities of the learner to adapt its teaching behavior. 

\paragraph{Active learners} 

The interested reader can refer to \cite{lopes2014active} for a review of active learning for autonomous intelligent agent. In the following paragraphs, we only focus on active learning agents in social interactive learning conditions. The notion of uncertainty is often used in active learning algorithm. Uncertainty refers to situation where the agent does not know how to behave in order to fulfill the task. By collecting more information about that situation, the agent should reduce uncertainty and increase its performance on the task.

% Active learning approaches endow the learner with the power to select which demonstrations the teacher should perform. 

Several criteria have been proposed: game theoretic approaches \cite{shon2007active}, entropy \cite{macl09airl,melo2010learning}, query by committee \cite{judah2012active}, membership queries \cite{melo2013multi}, maximum classifier uncertainty \cite{chernova09jair}, expected myopic gain \cite{cohn2010selecting,cohn2011comparing} and risk minimization \cite{doshi2008reinforcement}. Such ideas have been applied in situations as different as navigation \cite{macl09airl,cohn2010selecting,cohn2011comparing}, simulated car driving \cite{chernova09jair} or object manipulation \cite{macl09airl}.

A number of previously presented works already includes an active component to their agents. For example, in \cite{macl11simul}, the agent is more efficient at learning both the task and the meaning of new signals when seeking for uncertain state-action pairs. In \cite{judah2012active}, the authors consider active imitation learning. Instead of passively collecting demonstrations from the user, the learning agent queries the expert about the desired action at specific states.

In \cite{chernova09jair}, the authors propose to balance autonomy and demonstration request using a confidence estimate, measured by the uncertainty of the classifier. The robot asks for demonstration only in states it is unsure about what to do. Otherwise the robot acts autonomously but can still be corrected by the user at any time. A problem with this approach is that the information on the dynamics of the environment is not taken into account when learning the policy. To address this issue, Melo et al. \cite{melo2010learning} proposed a method that computes a kernel based on MDP metrics \cite{taylor2009bounding} that includes the information of the environment dynamics. In this way the topology of the dynamics is better preserved and thus better results can be obtained. They use the method proposed by Montesano et al. \cite{montesano2012active} to make queries where there is lower confidence of the estimated policy.

Directly under the inverse reinforcement learning formalism, one of the first approaches were proposed by Lopes et al. \cite{macl09airl}. After a set of demonstration has been observed, it is possible to compute the posterior distribution of reward that explain the teacher behavior. By seeing each sample of the posterior distribution as a different expert, the authors took a query by committee perspective allowing the learner to ask the teacher what is the correct action in the state where the predicted policies are more different. This work was latter extended by considering not just the uncertainty on the policy but the expected reduction in the global uncertainty \cite{cohn2010selecting,cohn2011comparing}. In a reinforcement learning framework, the learner could directly ask about the reward value at a given location \cite{regan2011eliciting} and it has been shown that reward queries can be combined with action queries \cite{melo2013multi}.

The previous works on active inverse reinforcement learning can be seen as a way to infer the preferences of the teacher. This problem of preference elicitation has been addressed in several domains \cite{furnkranz2010preference,chajewska2000making,braziunas2012local,viappiani2010optimal,brochu2010tutorial}.

% \cite{lopes2012strategic} strategic student metaphor: a student has to learn a number of topics (or tasks) to maximize its mean score, and has to choose strategically how to allocate its time among the topics and/or which learning method to use for a given topic. maximize learning gain is optimal. 


\paragraph{Active teachers}

An active teacher tries to provide demonstrations or instructions that will make the learning process more efficient for the learning agent. 

In \cite{cakmak2012algorithmic}, the authors study how a teacher can optimally provide demonstrations for a sequential problem. Concretely, the teacher should find the smallest sequence of examples that allow the learner to identify the task. Their optimal teaching algorithm allows a much faster convergence in all four presented tasks. Similarly in \cite{torrey2013teaching}, the teacher has a limited number of advises to give and the authors study how to best use these advises to improve the learning gain of the learning agent. They showed that advices could have greater impact when they are spent on important states, or to correct agent's mistakes.

Active teaching finds applications in several domains, especially in the educational one, where giving individual advises for each student given their individual proficiency may improve the collective learning gain of a classroom. For example, in \cite{clement2014online} the authors presents an \emph{intelligent tutoring systems} which \textit{``adaptively personalizes sequences of learning activities to maximize skills acquired by each student''}. They take into account constraints about the limited time and motivation resources of each student. Their approach seeks at optimizing the learning gain of students, by selecting the exercises that should make the student progress best.
 
\transition

In chapter~\ref{chapter:planning} we will present an active version of our algorithm. As for other works, our active learner will seek at reducing uncertainty by reaching states of maximal uncertainty. However, our uncertainty measure differs from previous works in that both the task and the signal to meaning mapping is unknown at start. Therefore there is uncertainty both at the task and at the signal level, which required developing a new uncertainty measure specific to our problem.

\subsection{Discussion}

In this section we discovered a number of works dealing with the human teacher inside an interaction loop. We have seen that information coming from a human teacher cannot always be considered as optimal or following simple mathematical rules. Moreover as each user is different, current research are advancing toward modeling the user teaching behavior during the interaction. Yet to model some aspects of the user, the robot is assumed to have access to an explicit known source of information about either the task or the meaning of some signals.

% will never have access to an explicit known source of information. The robot

In this thesis, we want to learn from unlabeled interaction frames. It means that the robot will not know the meaning of the signal it receives, neither the particular task it should achieve. However the robot is already equipped with a theoretical model of the human teacher, and is able to deduce the meaning the user should send given a specific context (state-action pair) and a specific task. Moreover the user is assumed to be consistent, i.e. a user behavioral model is provided to the robot.

Our two latter assumptions are conflicting with the observations about the behavior of human teachers presented in this section. To account for variability between users, we will simply introduce a noise parameter in our models. In chapter~\ref{chapter:limitations}, we soften the assumption that the robot is equipped with a theoretical model of the human teaching behavior.

Finally we will consider an active learning agent and present in chapter~\ref{chapter:planning} a new uncertainty measure that takes into account both the uncertainty about the task and the uncertainty about the signal to meaning mapping.

%%%%%%%%%%%%%%%%%%%%%%%%%%%%%%%%%%%%%%%%%%%%%%
%%%%%%%%%%%%%%%%%%%%%%%%%%%%%%%%%%%%%%%%%%%%%%
%%%%%%%%%%%%%%%%%%%%%%%%%%%%%%%%%%%%%%%%%%%%%%
%%%%%%%%%%%%%%%%%%%%%%%%%%%%%%%%%%%%%%%%%%%%%%
%%%%%%%%%%%%%%%%%%%%%%%%%%%%%%%%%%%%%%%%%%%%%%
\section{Language Acquisition}
\label{chapter:related:language}

While this is not the main target of this thesis, this work is also relevant with regards to the computational modeling of language acquisition. The general question of how certain sub-symbolic communication signals can be associated to their meanings through interaction has been largely studied in the literature. But the specific question of how teaching signals (e.g. speech words) can be mapped to teaching meanings, and how they can be used for learning new tasks, has, to our knowledge, not been computationally modeled.

The literature on the computational modeling of language acquisition by machines and robots is large and diverse, and focused on many aspects of language learning \cite{steels2012grounding,steels2002aibos, cangelosi2010integration, kaplan2008computational, steels2003evolving, brent1997computational, yu2007unified}. An important line of work investigated the Gavagai problem \cite{quine1964word}, i.e. the problem of how to guess the meaning of a new word when many hypothesis can be formed (out of a pointing gesture for example) and it is not possible to read the mind of the language teacher. Various approaches were used, such as constructivist and discriminative approaches based on social alignment \cite{steels06spatialLanguage, steels2008can}, pure statistical approaches through cross-situational learning \cite{xu2007word, smith2008infants} or more constrained statistical approaches \cite{roy2005semiotic, yu2007unified}. In all these existing models, meanings were expressed in terms of perceptual categories (e.g. in terms of shape, color, position, etc) \cite{steels06spatialLanguage, steels2008can,yu2007unified}, or in terms of motor actions \cite{steels2008robot, Massera2010,sugita05a}. This applies to models implemented in robots, such as in \cite{heckmann2009teaching}, where the robot ASIMO is taught to associate new spoken signals to visual object properties, both in noisy conditions and without the need for bootstrapping. 

\subsection{Language games}

The work of Steels and colleagues \cite{steels2012grounding,steels2002aibos} have extensively shown the importance of  language games, instantiating various families of pre-programmed interaction frames specifically designed to allow robots to learn speech sounds \cite{de2000self,oudeyer2006self}, lexicons \cite{steels2002aibos} or grammatical structures \cite{steels06spatialLanguage, steels2008can}. Other works used similar interaction protocols to allow a structured interaction between humans and robots so that new elements of language could be identified and learnt by the robot learner \cite{roy02a,lyon2012interactive,cangelosi06b,yu2004multimodal,cangelosi2010integration,sugita05a,dominey2005learning,cederborg2011imitating}. In particular, it was shown that these interaction protocols fostered efficient language learning by implementing joint attention and joint intentional understanding between the robot and the human \cite{kaplan2006challenges,yu2005role,yu2007unified}, for example leveraging the synchronies and contingencies between the speech and the action flow \cite{rohlfing2006can,schillingmann2011acoustic}.

Most of the existing models study communicative signals whose meanings were expressed in terms of proper names, color and shape terms, motor actions, or body postures. Only very few models so far have explored how other categories of word meanings could be learned. Cederborg et al. presented a model where word meanings expressed the cognitive operation of attentional focus \cite{cederborg2011imitating}. Some models of grammar acquisition dealt with the acquisition of grammatical markers which meaning operates on the disambiguation of other words in a sentence \cite{steels2012fluid}. Spranger et al. studied how a spatial vocabulary and the concepts expressed by it can emerge in a population of embodied agents from scratch. They considered the emergence of various spatial language systems, such as projective, absolute and proximal \cite{spranger2012emergent,spranger2013grounded}, of spatial relations, such as landmarks \cite{spranger2013evolutionary}, and of basic spatial categories such as left-right, front-back, far-near or north-south \cite{spranger2012co}. Finally, the Lingodroid project \cite{schulz2010robots} used robotic rats (called iRats) as embodied agent to study the emergence of geopersonal spacial language and language for time event (such as day-night cycle) in a population of robots. iRats were equipped with shared attention mechanism, they could measure the light level and they were able to build their own map of the environment. Pairs of robots could play a meet-at and meet-when game. By repetitively playing the game, the robots population agreed on specific terms for spacial communication and time of the day, such as the concept of morning or afternoon \cite{schulz2011lingodroids,heath2012long}. These concepts of morning and afternoon were changing with the season according to the lightning cycle and allowed robot to synchronize their behavior based on relative cyclic time rather than an absolute notion of time or a calendar.

Language games usually consider a direct relation between the communicative signals and the environment. For example, the agents learn to associate names to objects, colors, spatial relations, or time events. The problem considered in this thesis will consider more abstract relation between the communicative signals and their meaning, such as whether the past action of one agent was ``correct'' or ``incorrect'' with respect to a global objective. Or if the agent should have move ``left'' or ``right'' to get closer to the goal. While there is no specific limitation from our work to handle typical language game scenarios, most of the method presented above have not be applied to the more abstract relation considered in this thesis. Finally most of the works presented so far consider a rather rigid interaction protocol between agents, where the communication goal is often defined before hand. For example, when playing a meet-at or a meet-when game, the iRat robots are aware that the communicative signals respectively refer to a location on the map or to a time event as measured by their light sensors.

In the next subsection, we highlight the work of Cederborg et al. \cite{cederborg2011imitating} which, to our knowledge, is the closest work in language acquisition considering a setup similar to the problem of \emph{learning from unlabeled interaction frames}. 

% As they have phrased it, they \textit{``show that it is possible to simultaneously learn never before encountered communicative signs and never before encountered movements, without using labeled data, and at the same time learn new compositional associations between movements and signs''}.

\subsection{Work of Thomas Cederborg et al.}
\label{chapter:related:language:thomas}

In this subsection, we present the work of Thomas Cederborg as published in \cite{cederborg2011imitating,cederborg2013language} and in the chapter 6 of his thesis manuscript \cite{cederborg2014thesis}. This work has been categorized in the language acquisition field by the authors but it has wider application especially in human-machine interaction. As we will discuss in the following paragraphs, this work is strongly related with our problem of \emph{learning from unlabeled interaction frames} and the solution proposed to their problem is closely linked with the algorithm proposed in this thesis.

In \cite{cederborg2011imitating}, Cederborg et al. \textit{``show that it is possible to simultaneously learn never before encountered communicative signs and never before encountered movements, without using labeled data, and at the same time learn new compositional associations between movements and signs''}. They present an experiment where a robot learns to produce appropriate gestures in response to the communicative signals of one human, called an interactant. To do so, the robot can observe another human, called the demonstrator, which already knows how to interpret the interactant signals and produce the corresponding gestures. The interactant always provides two consecutive symbolic signals, one is associated to a type of gesture (e.g. drawing a triangle or a circle) and the other is associated to a drawing referential (e.g. red, blue or green object). The demonstrator, which knows how to interpret the interactant symbols, can then demonstrate the appropriate task, for example drawing a circle around the blue object. The robot observes both the interactant signals and the demonstrator trajectories and learns both the meaning of the communicative signals of the interactant and how to respond to them.

This setup is closely related with our problem of \emph{learning from unlabeled interaction frames} as both the task and the signal to meaning mapping are unknown at start. A number of differences can be listed: \begin{inparaenum}[a)] \item the robot is not active in the learning process and passively observes the interactant and the demonstrator, \item  the robot has access to full demonstrations of the task, and \item the association between the task and the signals is direct, whereas in the scenario considered in this thesis the meaning of the signals are more abstract and for example refer to whether the action was ``correct'' or ``incorrect'' with respect to the aimed task. \end{inparaenum} However their setup requires to learn the meaning of two symbolic communicative channels (type of gesture or drawing referential), as well as the particular signal to meaning mapping within each channel (triangle/circle and red/blue/green). The problems we tackle in this thesis only consider one channel of communication. In addition their agent can learn the gestures and generalize reproduction in other coordinate systems given previously unseen combination of interactant signals. In this thesis, we will also demonstrate how our agent can reuse their knowledge about the interactant signals to learn new tasks faster.

But the most interesting aspect of their work lies in the introduction of interpretation hypothesis. Even if not explicitly named that way in their early work \cite{cederborg2011imitating}, the terms of interpretation hypothesis was central to the thesis of Thomas Cederborg \cite{cederborg2014thesis} and it is also a central concept in the present thesis. An interpretation hypothesis is the fact of systematically interpreting or evaluating the observed data with respect to a set of hypotheses. In their work the hypothesis set corresponds to the referential of the demonstrated trajectories, unknown at start but known to belong to a finite set of possible referential (e.g. there is only three objects). By making the hypothesis that each trajectory refer to each of the referential (see Figure~5 of \cite{cederborg2011imitating}), they can find out which gesture belong to which referential and which trajectories are of the same type (see Figure~6 of \cite{cederborg2011imitating}). Similar ideas are pushed forward in this thesis, however we note that in the work of Cederborg et al. the agent was first grouping the trajectories per type and only then was able to identify the meaning of the communicative signals of the interactant. In our work, the process of learning the task is not differentiable from the process of learning the signal to meaning mapping.

We will summarize the similarities and differences between the work presented in this thesis and several works presented in this chapter in section~\ref{chapter:related:discussion}.

\subsection{Semiotic experiments} 

In this subsection, we briefly introduce the field of experimental semiotics, and briefly introduce our experimental scenario that study how human can deal with the problem of \emph{learning from unlabeled interaction frames}. More details will be provided in chapter~\ref{chapter:humanexperiment}.

The ability to learn from unlabeled interaction frames might seem to be an artificial and unrealistic scenario made up for practical purposes in human-machine interaction. Yet, this capability is crucial in infant social development and learning, as well as in adult mutual adaptation of social cues. This has been the subject of experiments in experimental semiotics \cite{galantucci2009experimental}. 

The field of experimental semiotics studies the emergence and evolution of communication systems \cite{galantucci2009experimental}. Instead of computer simulations as presented in previous subsections \cite{cangelosi2002simulating,steels2012experiments}, controlled experiments in laboratory settings are designed to observe communication between human participants who perform joint tasks. For instance, Galantucci et al. showed that pairs of participants performing a joint task could coordinate their behaviors by agreeing on a symbol system \cite{galantucci2005experimental}.

Most experimental semiotics studies developed to study joint action involve symmetric communication (cf. \cite{Galantucci2011experimental}), where both participants are able to send and receive communicative signals. In this thesis, we study asymmetric communication where only one of the two partners can send signals. To our knowledge two semiotic studies have considered asymmetric communication \cite{de2010exploring,griffiths2012bottom}. 

The work conducted by Griffiths et al. \cite{griffiths2012bottom} is more directly related to our problem of \emph{learning from unlabeled interaction frames}. They explore a human-to-human interaction in a categorization task where instructions can only be provided via six unlabeled symbols (thus the meaning of teaching signals are unknown to the learner). The learner has however access to some environmental reward on its performance on the task. This study shows that tutors seem to spontaneously use three main types of instruction in order to help the learner: positive feedback, negative feedback, and concrete instructions (e.g. name of next optimal action).

In chapter~\ref{chapter:humanexperiment}, we will present our experiment setup which is a variant of the work of Griffiths et al., where teaching signals are unknown at start, sub-symbolic and not from a pre-determined set. However in our experimental scenario it is impossible for the learner to perform the task without understanding the communicative acts of the teacher. By removing access to an environmental reward to the participants, the learner is no more able to improve its understanding of the task independently of the understanding of the teaching signals; which makes our experiment more suited to study how humans deal with the problem of \emph{learning from unlabeled interaction frames}. Astonishingly, even with such unconstrained interaction, we will see that most participants agreed on a communication system and succeeded in solving the task.

%%%%%%%%%%%%%%%%%%%%%%%%%%%%%%%%%%%%%%%%%%%%%%
%%%%%%%%%%%%%%%%%%%%%%%%%%%%%%%%%%%%%%%%%%%%%%
%%%%%%%%%%%%%%%%%%%%%%%%%%%%%%%%%%%%%%%%%%%%%%
%%%%%%%%%%%%%%%%%%%%%%%%%%%%%%%%%%%%%%%%%%%%%%
%%%%%%%%%%%%%%%%%%%%%%%%%%%%%%%%%%%%%%%%%%%%%%
\section{Multi-agent interaction without pre-coordination}

As robots are moving into the real world, they will increasingly need to group together for cooperative activities with previously unknown teammates. In such ad hoc team settings, team strategies cannot be developed a priori. Rather, each robot must be prepared to cooperate with many types of teammates, which may not share the same capabilities or communicative means. This challenge of multi-agent interaction without pre-coordination (MIPC), also called the pickup team challenge \cite{gil2006dynamically} or the ad-hoc team challenge \cite{stone2010ad}, states that agents should learn to collaborate without defining pre-coordination schemes and/or without knowing what the other agents will be capable of \cite{bowling2005coordination,gil2006dynamically,stone2010ad}. The ad-hoc team challenge is specific to scenarios where one agent is removed from a working and synchronized team, and replaced by a new agent, called the ad-hoc agent, which never interacted with the team before \cite{stone2010ad}.

A prototypical example is the one of a street soccer team. Such team is composed of players coming from different areas of a city, with different soccer skills, different preferences in terms of placement on the field, and even different ways of communicating game strategies. Yet such teams are quickly formed and functional in a matter of minutes. MIPC aims at creating agents solving similar problems. Among others, researchers in the field have considered soccer teams scenarios\cite{bowling2005coordination}, treasure hunting tasks \cite{gil2006dynamically}, bandit problems \cite{barrett2013communicating}, and the pursuit domain \cite{barrett2011empirical}.

This area of research is still in its early stages and the full challenge of MIPC is difficult to tackle directly. Researchers have started investigated only certain aspects of the larger problem by making suitable assumptions. The most common assumption is that all agents on the field share a common objective, i.e. that all agents are all partners towards achieving the same task \cite{barrett2011empirical}. In \cite{bowling2005coordination,gil2006dynamically} all agents follow complex pre-specified plans where each agent can be attributed a role to which is associated synchronized action sequences. In \cite{stone2010teach,stone2013teaching}, the ad-hoc agent knows the behaviors of the other agents and assume it is fixed (i.e. other agents do not learn). 

There are different roles an ad-hoc agent can play in the team:

\begin{itemize}

\item A first scenario is when the new agent knows the environment and the task to achieve. In this case, the ad-hoc agent must influence the other agents to achieve the correct task. For example, in  \cite{stone2010teach,stone2013teaching}, an ad-hoc agent should influence other agents' behaviors such that the team gets more payoffs or to guide the other agents towards specific states. This ad-hoc agent cannot communicate directly with the other agents. However the other agents' behaviors are known and are influenced by the ad-hoc agent actions. The problem is therefore to find the correct sequence of actions that may lead the other agents towards the correct states, resulting in a higher performance on the task. 

\item A second scenario considers that all agents share the same goal, but the new ad-hoc agent does not know a priori the behaviors of its partners. To help solving the task, the ad-hoc agent should learn other agents' behaviors and selects its actions accordingly  \cite{barrett2011adhoc,barrett2011empirical,barrett2013team}. For example, in \cite{barrett2011empirical} the ad-hoc agent should help its teammates catch a prey and is more efficient when trying to understand the behavior of the other agents. Often to make this problem feasible, it is assumed that the other agents sample their latent policy (or type) from a finite set. The ad-hoc agent then only has to learn to match each agent with its true model. In \cite{albrecht2014uai}, the authors analyzed convergence properties of this kind of scenario. But sometimes, the other agents are totally unknown to the ad-hoc agent. For example, in \cite{barrett2011empirical} the ad-hoc agent models online and from scratch the behavior of its teammates. Even for cases when students, on which the authors had no control, have designed the other agents, the algorithm of the authors was able to perform even better than the initial student teams.

\end{itemize}

Finally, it is only recently that explicit, but initially unknown, communication between agents has been considered. Samuel Barrett et al. introduced an abstract arm bandit domain with communication \cite{barrett2013communicating}. This work is, to our knowledge, the first work in MIPC considering communication between agents and where the ad-hoc agent initially does not know how the other agents interpret its messages. However this problem differs from the challenge of \emph{learning from unlabeled interaction frames} as the task the agent should optimize could be inferred without the use of communication through environmental reward only, and communication only intends to speed up the learning process.

\transition

Some aspects of MIPC are closely related to our problem of learning from unlabeled interaction frames, such as the challenge of communication between teammates. Considering robots can come from different factories in different countries, they might not use the same protocols of interaction and adapting to such protocols is a central future challenge of MIPC. Yet, the communication aspect has been only little investigated \cite{barrett2013communicating}, and we believe the work presented in this thesis can bring interesting perspectives to the MIPC challenge. Especially it can be interesting to investigate domains where communication between agents is mandatory to succeed in the task, but where communication protocols between teammates are a priori unknown.

%%%%%%%%%%%%%%%%%%%%%%%%%%%%%%%%%%%%%%%%%%%%%%
%%%%%%%%%%%%%%%%%%%%%%%%%%%%%%%%%%%%%%%%%%%%%%
%%%%%%%%%%%%%%%%%%%%%%%%%%%%%%%%%%%%%%%%%%%%%%
%%%%%%%%%%%%%%%%%%%%%%%%%%%%%%%%%%%%%%%%%%%%%%
%%%%%%%%%%%%%%%%%%%%%%%%%%%%%%%%%%%%%%%%%%%%%%
\section{Unsupervised learning}

Unsupervised learning is the problem of finding hidden structures in unlabeled data. It mostly applies in clustering tasks where a dataset is divided into subgroups of data sharing similar characteristics, such as a close proximity in the feature space. In the following, we present two unsupervised learning problems that share some similarities with our problem of \emph{learning from unlabeled interaction frames}.

% Even if there is no explicit error or reward to evaluate a potential solution, there is still predefined metrics which define what the terms \emph{structure} means. Finding an hidden structure is more looking from known patterns in unlabeled data.

\paragraph{Unsupervised multimodal learning} In unsupervised multimodal learning, the system has access to synchronized raw information from multiple modalities. A particular instance of multimodal learning is the acquisition of language where the learner has to link perception of an object to the sound of its name, or of a sound to a gesture such as in \cite{mangin2013learning}. The learner receives continuously a visual and an audio stream and should learn to associate parts of the visual information with their associated audio stimulus. But the visual and audio informations are already synchronized such that the relevant information from the visual stream is perceived simultaneously with its associated audio stimuli.

In a robotic application, Yasser Mohammad et al. used multimodal learning to segment and associated gesture commands from a user to actions of a robot \cite{mohammad2009unsupervised}. The gestures and actions were observed from a continuous stream extracted from a Wizard of Oz experiment (where the robot is secretly controlled by a human). They relied on a motif discovery algorithm to identify recurrent and co-occurrent patterns in the gesture and action flow \cite{mohammad2009constrained}. In \cite{mohammad2010learning} the same authors extended their approach to allow their system to derive controllers for the robot and not just find recurrent patterns, as well as a methods to accumulate the acquired knowledge for long term operation.

However, while being unsupervised, the stream of data where synchronized and collected using a Wizard of Oz setup, meaning that the association between the gestures and the robot's actions was provided. And importantly, the relation between the gesture commands from the user and the actions of the robot was direct. Contrary to our problem of learning form unlabeled interaction frame, there is no intermediate steps of analysis required to infer the meaning of the human gestures.

\paragraph{Simultaneous localization and mapping}

Simultaneous localization and mapping (SLAM) \cite{smith1990estimating,dissanayake2001solution} is  the problem of constructing a map of an unknown environment while simultaneously keeping track of the robot's location in that environment. 

SLAM seems to include a chicken and egg problem. To build the map, the robot needs to know its location on the map such as to be able to include its current measurements to the map. And to know its location on the map, the robot needs to know the map such as to infer its position from its measurements. In practice, the answers to the two questions cannot be delivered independently of each other.

However the robot knowns that the data received from its sensors refers, for example, to noisy information about distances to obstacles. The robot also often knows the qualities of its sensors and motors, and roughly how it's actions influence its position. For example, by measuring changes in wheels rotary encoders, the robot can approximate its position shift after small control commands. Accessing to an approximation on its position shift, the robot can now update the map given its new sensory information. Using only this source of information is limiting, especially because every error accumulates over time. There are several others sources of information the robot can rely on. For example, the environment is often assumed to be fixed. Hence the robot can track its relative position to some landmarks, and incrementally update its position on the map while detecting some other landmarks and incrementally building the map.

\transition

Unsupervised learning also deals with unlabeled data. But contrary to our problem, unsupervised learning only identifies direct relations between observations. In our problem of \emph{learning from unlabeled interaction frames} the system must also identify a task, unknown at start, from the incoming unlabeled data. This makes the relation between observations non direct. Indeed, the association between the different observations requires an additional abstract piece of knowledge, i.e. the task, that is yet unknown at the beginning of the interaction.

\section{Brain computer interfaces}

EEG-based brain-computer interfaces (BCIs) have been used successfully to control different devices, such as robotic arms and simulated agents, using self-generated (e.g. motor imagery) and event-related potentials signals (see \cite{millan10} for a review). Error-related potentials (ERPs) are one kind of event-related potential appearing when the user's expectation diverges from the actual outcome \cite{Falkenstein00,chavarriaga2014errare}. Recently, they have been used as feedback instructions for devices to solve a user's intended task \cite{chavarriaga2010learning,iturrate13}.

As in most BCI applications, ERP-based BCI requires a calibration phase to learn a decoder (e.g. a classifier) that translates raw EEG signals from the brain of each user into meaningful instructions. This calibration is required due to specific characteristics of the EEG signals: non-stationary nature \cite{vidaurre11}, large intra- and inter-subject variability \cite{Polich1997}, and variations induced by the task \cite{iturrate2013task}. The presence of an explicit calibration phase, whose length and frequency is hard to tune and is often tedious and impractical for users, hinders the deployments of BCI applications out of the lab. 

Thus, calibration free methods are an important step to apply this technology in real applications \cite{millan10}. We note that the problem of \emph{learning from unlabeled interaction frames}, which is central to this thesis, is the same problem as removing the calibration procedure for interactive systems, whose BCI is a good example. Despite the importance of calibration-free BCI, there are only few BCI applications that are able to calibrate themselves during operation.

Several works considered online adaption of classifiers. In \cite{vidaurre2010towards} the authors show that it is possible to adapt the decoder online for long term operation using sensory-motor rhythms. Similarly for BCI based on
event-related potentials or steady-state evoked potential (SSEP) many works have studied how to continuously adapt the brain decoder \cite{fazli2009subject,lu2009unsupervised,fazli2011l1,congedo2013new,schettini2014self}.

However, while the above methods allow a more flexible and online adaptation to each user, they are not strictly calibration-free methods. They require a relatively smart prior on the decoder of brain signals beforehand. Such prior is usually extracted from intersubject information \cite{fazli2009subject,lu2009unsupervised,vidaurre2010towards}. We identified two other works that start the adaptation process from a randomly seeded classifier. While still requiring a prior on the classifier these methods have been shown to be robust to a large range of initialization.

In invasive BCI, Orsborn et al. proposed a method to learn from scratch and in closed loop a decoder for known targets using pre-defined policies to each target \cite{Orsborn2012}. However, their method requires a warm-up period of around 15 minutes. Using non-invasive technologies (EEG based), to our knowledge only one group of researchers achieved calibration-free interaction \cite{Kindermans2012a,kindermans2014true}. We detail their work in the following subsection.

\subsection{Work of Pieter-Jan Kindermans et al.}
\label{chapter:related:bci:kindermans}

Kindermans et al. considers the problem of P300 spellers. A P300 signal is an event-related potential elicited in the process of decision making \cite{polich2003theoretical}. It is evoked by the reaction to a visual or auditory stimulus, and it is linked with the process of evaluation or categorization of stimulus by our brain. 

A P300 speller exploits the properties of P300 ERPs to build a communication tool allowing users to input texts or commands to a computer by thought. The speller interface consists of letters arranged in rows and columns (see Figure~\ref{fig:speller}). The user is asked to focus his sight on the letter he wants to write. Then the rows and columns of the matrix are successively and randomly highlighted. By detecting the P300 signals in the users brain activity, it is possible to decode which row and column are associated to the letter the user wants to write. As each rows an columns are flashed the same number of times, the P300 stimulus as a frequency of $\frac{1}{N}$ (where $N$ is the number of rows or columns of the matrix).

\begin{figure}[!htbp]
  \centering
  \includegraphics[width=0.3\columnwidth]{\imgpath/speller.png}
  \caption{A speller interface with the third row highlighted.}
  \label{fig:speller}
\end{figure}

Kindermans et al. proposed a method to auto-calibrate the decoder of P300 signals by exploiting multiple source of information \cite{kindermans2012b,kindermans2014integrating}. As for most of the work presented above, they consider transfer learning where a model of previous subjects is used to \textit{``regularizes the subject-specific solution towards the general model''}. As it is a spelling task, they also make use of language models as a prior probability on the possible next letter. They also include a dynamic stopping criterion that is a measure of confidence on the next letter allowing the system to stop when it reaches a confidence threshold. Finally, and of more interest for us, they make use of unsupervised learning using an EM algorithm to update the classifier as new data comes in. They exploit the particular fact that among the multiple stimulations only one event out of six encodes a P300 potential in the speller paradigm.

While still requiring to bootstrap the system with several random classifiers as well as a warm-up period, Kindermans et al. have shown their unsupervised learning method coupled with specific properties of the task allows to start interacting with a speller without the need for calibration procedure \cite{Kindermans2012a,kindermans2014true}. This achievement correspond to solving the problem of \emph{learning from unlabeled interaction frame} and is therefore of high interest for our work. We now explain what specific information was used to solve this problem and identify it as being of a very specific nature, which differs from all other approaches.

As detailed earlier, the P300 speller problem offers some guarantee on the repartition of ``correct'' and ``incorrect'' P300 events. Only one row and one column should elicit a P300 response. In the case of a 6 rows speller, if each row are systematically scanned the same number of time, only one signal out of 6 will encode a positive P300 signal. And even more informative is the fact that, even if the wrong letter is identified in the end, at least 4 labels out of 6 will be correctly assigned. Indeed, if the wrong letter is identified, two labels will be swapped, resulting in two association errors, but still four ``incorrect'' labels will be correctly assigned. Obviously, if the correct letter is identified, the ``correct'' label will be correctly assigned, as well as the five ``incorrect'' labels. In the end, this is quite a lot of information that can offer good guarantees for their EM algorithm to identify properly the ``incorrect'' signal cluster; leaving the second cluster for the ``correct'' signals. As more data are collected, the EM algorithm will be better at identifying the underlying structure of the data and will be able to identify the cluster of ``correct'' signals from the one of ``incorrect'' signals given the constraints detailed above. As the process continues, identifying further letter is made easier, and importantly, by going back in the history of interaction, the system can correct letters that were wrongly identified.

As we will discover in next chapters, our method to do requires having access to such constraints and guarantees about the task, which makes our work easily generalizable to many types of problems. However, the work of Kindermans et al. already exploits information of a very specific nature to solve the problem of \emph{learning form unlabeled interaction frames}. Contrary to all the other approaches, their information source does not provide a direct knowledge about the task (as a language models do), neither about how to decode the signals themselves (as transfer learning methods do). It rather provides information emerging for the joint combination of a task and of a signal decoder. That is, that for the correct task (i.e. the correct letter), only one signal should be classified as ``correct'' and all the others as ``incorrect''. 

\transition

This type of information, that acts neither on the task, neither on the signal decoder, but rather on the combination of both is at the core of the work we will present in forthcoming chapters. As we have seen in section~\ref{chapter:related:language:thomas}, Cederborg et al. also make us of a similar source of information but reasoning about the consistency of some gestures with respect to different geographical references, e.g. object positions. We will summarize those works in next section~\ref{chapter:related:discussion} and highlight the differences and improvements of our method.

\section{Discussion}
\label{chapter:related:discussion}

We reviewed an extensive number of related works ranging from the computational modeling of language to more practical brain computer interaction problems. While releasing some important assumptions on the interaction, in most of those works the communicative signals had a direct relation to one element of the environment or to the task itself, such as being the name of a color, a shape, or a gesture type. In our work the signal to meaning relation will be more abstract such as whether an action was ``correct'' or ``incorrect'' with respect to an objective. Also, in most of these existing works the interaction between partners was pre-programmed and most of the time the robot knew how to use or understand communicative signals innately, e.g. how the teacher expresses ``correct'' or ``incorrect'' feedback. 


% . We allow for some ``uniform'' teaching mistakes but the teacher are assumed to not suffer from systematic errors or bias. Hence, we will

We note that in this thesis we will assume teachers are optimal and simply model some percentage of teaching mistakes to account for the variability between users. This might not be an accurate assumption given the work presented in the beginning of this chapter about human teaching behaviors. However our method is not restricted to the use of optimal teacher models, the only requirement is to have access to model of the human teaching behavior, which may include systematic errors or bias.

The work we present in this thesis shows mechanisms allowing a learner to simultaneously learn a new task and acquire the meaning associated to feedback and guidance signals in the context of social interaction. Furthermore, we show mechanisms allowing the learner to leverage learned signals' meanings to acquire novel tasks faster from a human. To our knowledge, only two works are tackling the same problem as the one presented in this thesis. And surprisingly, those two works lies in the computational modeling of language acquisition (work of Cederborg et al. in section~\ref{chapter:related:language:thomas}) and in the BCI domain (work of Kindermans et al. in section~\ref{chapter:related:bci:kindermans}). 

Especially, it is in the BCI domain that the idea of adaptive interface seems to be highly developed, with many methods to continuously adapt a brain decoder during operation. This may be explained by the specific nature of brain signals, which are not a natural way for humans to interact with machines. Therefore humans do not share common abilities in their generation and use of brain signals, and at design time we cannot use our daily intuition for creating universal decoders of brain signals. This differs from work on speech or facial expression recognition where many a priori knowledge can be included into the system. This kind of consideration may explain why the problem of adaptive interfaces and our specific problem of \emph{learning form unlabeled interaction frame}  has only been considered recently in human-robot interaction scenarios.

In the following of this discussion we summarize the main similarities and differences between our work and the work of Cederborg et al. and of Kindermans et al. as respectively discussed in  section~\ref{chapter:related:language:thomas} and section~\ref{chapter:related:bci:kindermans}. For the interested readers, this discussion section may be worth reading again once the reader has been through the remaining of this thesis, especially through chapter~\ref{chapter:lfui}.

We can list a number of differences between the work presented in this thesis and the related work presented in this chapter:

\begin{itemize}

%%%%%%
%%%%%%
\item First, we explicitly define and provide some solutions to the problem of \emph{learning from unlabeled interaction frames}. This problem is still relatively new in the domain of human-machine interaction. It represents a new step towards creating machines able to flexibly adapt to each particular users by learning the way such users communicate specific meanings to the machine.

%%%%%%
%%%%%%
\item Compared to the work of Cederborg et al. \cite{cederborg2011imitating}, our robot is already equipped with sufficient skills to perform the task, i.e. if it knew the goal it could fulfill it by its own mean. In most of our experiment, the robot further knows that the task belong to a limited set of task. In \cite{cederborg2011imitating}, less constraints are applied on the task space, the robot only knows it will have to reproduce a continuous gesture of unknown type which is not restricted to belong to a limited set. However, in their work, one commmunicative channel directly encodes the ``name'' of the gesture demonstrated; in our work the relation between the teaching signals and the robot's actions is indirect and depend on the true unknown task.

%%%%%%
%%%%%%
\item Compared to the work of Kindermans et al. \cite{Kindermans2012a,kindermans2014integrating} our method does not require to bootstrap the system with random classifiers which are updated step by step but unreliable at start. Our method rather identifies the classifier from scratch. This difference is mainly due to the experimental setup used in our respective work. For example, in the P300 speller of Kindermans et al. a new letter must be identified every 15 flashes. Logically the system requires a warm up period that produces a high number of spelling errors in the beginning of each experiment. Such errors are however detected and corrected later on, after their so called ``eureka'' moment \cite{Kindermans2012a}, when their EM algorithm had access to enough data to identify the positive and negative clusters. To the contrary, by applying our method to the speller paradigm, the system would only pick a letter once it is confident that the letter is the correct one; therefore reducing dramatically the number of spelling errors but with a longer ``blank sheet'' period for the user in the beginning. However the computational cost of our method increases with the number of possible tasks (e.g. the number of rows and columns of the speller), which is not the case for the work of Kindermans et al..

\item Another difference between our work and the work of Kindermans et al. lies in the properties that their world should hold in order to ensure a proper functioning of their algorithm. In the work of Kindermans et al., the world should guarantee a specific ratio of ``correct'' and ``incorrect'' signals in the received signals. This ratio could be in favor of either one or the other label but is mandatory to be asymmetric, with more signals from one class than from the other. Indeed, their EM algorithm alone can identify two clusters in the feature space of the signal, but cannot attribute labels to each cluster without having access to additional information (the ratio of positive and negative P300 signals in their case). Our method is more generic and can be applied to a majority of sequential problems, even when it is impossible to define a sequence of actions that guarantee a specific ratio of meanings in the received signals. 

%Our algorithm is able to identify the correct task while simultaneously assigning the correct labels to the received teaching signals.

%%%%%%
%%%%%%
\item Compared to both Cederborg et al. and Kindermans et al. our approach is more generic and can be applied directly to a variety of sequential problems which are common in the human-robot and human-computer interaction domains. In particular we highlight the chicken an egg problem inherent to interacting with machine, and define the general challenge of \emph{learning from unlabeld intraction frames}. However, we note that this thesis focus on a very specific problem and more broad considerations are highlighted in the thesis of Thomas Cederborg \cite{cederborg2014thesis}.

%%%%%%
%%%%%%
\item We consider sequential task, which are tasks requiring the agent to perform a series of correct actions in order to fulfill the task correctly. Therefore there is a planning aspect involved which was not present in the work of Cederborg et al. where the robot passively observed interactant-demonstrator interactions, neither in the work of Kindermans et al. where the row and column flashes patterns were determined in advance. We note that the problem of P300 spellers used by Kindermans et al. could be represented as a sequential problem, where flashing a particular row or column represents the agent's available actions. However, if the sequence of actions is no more pre-defined, i.e. with the same number of flashes per row or column, the guarantees that only one signal out of $N$ encodes a positive ERPs would not be satisfied and their algorithm would be more likely to converge to a wrong classifier.

%%%%%%
%%%%%%
\item Given the sequential nature of our problems, we consider active learning which is the ability of our agent to actively selects its actions in order to improve its performance. As stated previously, this planning aspect was not considered in the work of Cederborg et al. and Kindermans et al.. We will show in chapter~\ref{chapter:planning} that planning when both the task and the signal to meaning mapping is unknown requires to develop a new measure of uncertainty. Our measure takes into account the uncertainty on both the task and the decoder; and is an important contribution of our work.

%%%%%%
%%%%%%
\item We also provide a number of extensions in chapter~\ref{chapter:limitations} to our algorithm, such as to cope for continuous state spaces and continuous task spaces. We further release the assumption that the interaction frame (either feedback or guidance frame) is known in advance and assume it belongs to a pre-defined set of possible interaction frames.

%%%%%%
%%%%%%
\item Moreover, aside from many empirical demonstrations in both simulated and real experiments, we also present in chapter~\ref{chapter:limitations:proof} a simple mathematical proof providing some guarantees on our method. To our knowledge, we provide the first proof showing that a system is able to learn simultaneously a task from human instructions as well as the signal to meaning mapping of the user's instruction signals.

%%%%%%
%%%%%%
\item Finally in chapter~\ref{chapter:bci}, we will test our algorithm in a BCI application. Our experiment differs from the one of Kindermans et al. \cite{Kindermans2012a,kindermans2014true} because our task is a target reaching task where the agent decides on its own which action to take next. This task is sequential, meaning that several actions must be executed to reach the goal. In addition, we use a different kind of error related potential signals to encode a ``correct'' or ``incorrect'' feedback for the agent. Our signal is of similar nature than the P300 signals used by Kindermans et al., i.e. they encode a binary event, however they are slower to elicit and are known to be harder to detect \cite{chavarriaga2014errare}.

\end{itemize}

Despite the differences between our work and the work of Cederborg et al. and Kindermans et al., there is similar fundamental properties of the problem that are exploited by our respective works. Especially the notion of interpretation hypothesis developed in Thomas Cederborg's thesis and the use of an information source that emerges only from a combination of constraints on the task and signal spaces. 

In \cite{cederborg2011imitating}, Cederborg et al. reasoned about the consistency of some gestures with respect to different geographical references, e.g. object position, knowing that the signals of the user could refer to only three possible coordinate systems and therefore relying on interpretation hypothesis. In \cite{Kindermans2012a,kindermans2014true}, Kindermans et al. reasoned about the ratio of positive and negative P300 ERP signals that should be observed for the correct letter. In our work, we propose to capture the coherence between the organization of the teaching signals in their feature space and their associated labels. We make use of interpretation hypothesis to create one set of signal-label pairs for each task. The correct task hypothesis is the one from which a more coherent, consistent, signal to meaning model emerges from the hypothetic labeling process. That way both the task and the signal to meaning model can be identified. Hence the assumption of coherence between the user behavior and our user model is a primordial prerequisite for our algorithm to work. Interestingly, this measure is more general than the one used by Kindermans et al. and does not require a specific ratio of  ``correct'' and ``incorrect'' signals to work.

As we will explore in the following chapters, this type of information, that acts neither on the task, nor on the signal decoder, but rather emerges from the combination of constraints on both task and signal spaces are fundamental properties we will exploit to solve the problem of \emph{learning from unlabeled interaction frames}.

\transition

Before presenting the core principles of our algorithm in chapter~\ref{chapter:lfui}, we present in next chapter (chapter~\ref{chapter:humanexperiment}) a semiotic experiment where two human partners must handle a similar situation than our problem of \emph{learning from unlabeled interaction frames}.

