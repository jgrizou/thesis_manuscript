%!TEX root = ../../thesis.tex
\renewcommand{\chapterpath}{relatedwork}
\renewcommand{\imgpath}{\chapterpath/img}

\chapter{Related Work}
\label{chapter:relatedwork}
\minitoc


Why: So that audience can implement their own version of the algorithm. It means they should be convince it is a useful algorithm, known when to use it, be aware of the many assumptions and understand the underlying principle. 

There is only one simple idea to understand for the algorithm and one for the planning. I want to illustrate them using visual example. It should be the same all along the thesis. The T world seem the simplest one that shows the main element for feedback instructions. for the guidance it may be a bit more difficult.

Who: Specialist, the two reviewers that will read the thesis and PY and Manuel so they let it go. No self-estime here. I will use the term we in most of the thesis cause the work is a common reflexion from people in and around the lab.

What: intro and related work cause it is necessary for the "who" guys. Brief motivation from the human experiment, then visual explanation of the problem, first a dummy symbolic one with maybe a proof followed by a 2D signal feedback one, then xp robot pick and place, then explanation of the planning problem + visual, then BCI example + prior information on the power, then limitation and example of what can be done to overcome them, then conclusion simple concise, no speculation on the future of robotic land.

When: End of June for a first advanced draft

Where: Latex, as short as possible, be concise

%%%%%%%%%%%%%%%%%%%%%%%%%%%%%%%%%%%%%%%%%%%%%%
%%%%%%%%%%%%%%%%%%%%%%%%%%%%%%%%%%%%%%%%%%%%%%
%%%%%%%%%%%%%%%%%%%%%%%%%%%%%%%%%%%%%%%%%%%%%%
%%%%%%%%%%%%%%%%%%%%%%%%%%%%%%%%%%%%%%%%%%%%%%
%%%%%%%%%%%%%%%%%%%%%%%%%%%%%%%%%%%%%%%%%%%%%%
\section{Interactive Learning}


%%%%%%%%%%%%%%%%%%%%%%%%%%%%%%%%%%%%%%%%%%%%%%
%%%%%%%%%%%%%%%%%%%%%%%%%%%%%%%%%%%%%%%%%%%%%%
%%%%%%%%%%%%%%%%%%%%%%%%%%%%%%%%%%%%%%%%%%%%%%
%%%%%%%%%%%%%%%%%%%%%%%%%%%%%%%%%%%%%%%%%%%%%%
%%%%%%%%%%%%%%%%%%%%%%%%%%%%%%%%%%%%%%%%%%%%%%
\section{Language games}


%%%%%%%%%%%%%%%%%%%%%%%%%%%%%%%%%%%%%%%%%%%%%%
%%%%%%%%%%%%%%%%%%%%%%%%%%%%%%%%%%%%%%%%%%%%%%
%%%%%%%%%%%%%%%%%%%%%%%%%%%%%%%%%%%%%%%%%%%%%%
%%%%%%%%%%%%%%%%%%%%%%%%%%%%%%%%%%%%%%%%%%%%%%
%%%%%%%%%%%%%%%%%%%%%%%%%%%%%%%%%%%%%%%%%%%%%%
\section{Unsupervised learning}

\subsection{Multimodal learning}


%%%%%%%%%%%%%%%%%%%%%%%%%%%%%%%%%%%%%%%%%%%%%%
%%%%%%%%%%%%%%%%%%%%%%%%%%%%%%%%%%%%%%%%%%%%%%
%%%%%%%%%%%%%%%%%%%%%%%%%%%%%%%%%%%%%%%%%%%%%%
%%%%%%%%%%%%%%%%%%%%%%%%%%%%%%%%%%%%%%%%%%%%%%
%%%%%%%%%%%%%%%%%%%%%%%%%%%%%%%%%%%%%%%%%%%%%%
\section{Ad hoc team}


%%%%%%%%%%%%%%%%%%%%%%%%%%%%%%%%%%%%%%%%%%%%%%
%%%%%%%%%%%%%%%%%%%%%%%%%%%%%%%%%%%%%%%%%%%%%%
%%%%%%%%%%%%%%%%%%%%%%%%%%%%%%%%%%%%%%%%%%%%%%
%%%%%%%%%%%%%%%%%%%%%%%%%%%%%%%%%%%%%%%%%%%%%%
%%%%%%%%%%%%%%%%%%%%%%%%%%%%%%%%%%%%%%%%%%%%%%
\section{The human in the loop}
