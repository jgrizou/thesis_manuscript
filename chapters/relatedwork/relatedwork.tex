%!TEX root = ../../thesis.tex
\renewcommand{\chapterpath}{relatedwork}
\renewcommand{\imgpath}{\chapterpath/img}

\chapter{Related Work}
\label{chapter:relatedwork}
\minitoc


Why: So that audience can implement their own version of the algorithm. It means they should be convince it is a useful algorithm, known when to use it, be aware of the many assumptions and understand the underlying principle. 

There is only one simple idea to understand for the algorithm and one for the planning. I want to illustrate them using visual example. It should be the same all along the thesis. The T world seem the simplest one that shows the main element for feedback instructions. for the guidance it may be a bit more difficult.

Who: Specialist, the two reviewers that will read the thesis and PY and Manuel so they let it go. No self-estime here. I will use the term we in most of the thesis cause the work is a common reflexion from people in and around the lab.

What: intro and related work cause it is necessary for the "who" guys. Brief motivation from the human experiment, then visual explanation of the problem, first a dummy symbolic one with maybe a proof followed by a 2D signal feedback one, then xp robot pick and place, then explanation of the planning problem + visual, then BCI example + prior information on the power, then limitation and example of what can be done to overcome them, then conclusion simple concise, no speculation on the future of robotic land.

When: End of June for a first advanced draft

Where: Latex, as short as possible, be concise

%%%%%%%%%%%%%%%%%%%%%%%%%%%%%%%%%%%%%%%%%%%%%%
%%%%%%%%%%%%%%%%%%%%%%%%%%%%%%%%%%%%%%%%%%%%%%
%%%%%%%%%%%%%%%%%%%%%%%%%%%%%%%%%%%%%%%%%%%%%%
%%%%%%%%%%%%%%%%%%%%%%%%%%%%%%%%%%%%%%%%%%%%%%
%%%%%%%%%%%%%%%%%%%%%%%%%%%%%%%%%%%%%%%%%%%%%%
\section{Interactive Learning}

\cite{macl11simul}, which presented a preliminary approach to this problem considering an abstract symbolic space of teaching signals in simulation and requiring to bootstrap the system with known symbols. 

Some other works have focused on learning semantic parsers, either from natural language as text \cite{branavan2011learning,kim2012unsupervised} or real speech \cite{doshi2008spoken}. Semantic parsers allow for a more natural human-robot interaction where more advanced set of instructions can be used. While in \cite{kim2012unsupervised} the algorithm can produce, with some limitation, previously unseen meaning representation, those works assume the agent has access to a known and constrained source of information at one stage. Either a direct access to its performances \cite{branavan2011learning}; to a reward from a teacher \cite{doshi2008spoken} or to a tuple (text instruction sentence, state, action sequence) where the instruction describes at a higher level the observed action sequence \cite{kim2012unsupervised}.

%%%%%%%%%%%%%%%%%%%%%%%%%%%%%%%%%%%%%%%%%%%%%%
%%%%%%%%%%%%%%%%%%%%%%%%%%%%%%%%%%%%%%%%%%%%%%
%%%%%%%%%%%%%%%%%%%%%%%%%%%%%%%%%%%%%%%%%%%%%%
%%%%%%%%%%%%%%%%%%%%%%%%%%%%%%%%%%%%%%%%%%%%%%
%%%%%%%%%%%%%%%%%%%%%%%%%%%%%%%%%%%%%%%%%%%%%%
\section{Language Acquisition}
\label{chapter:related:language}

While this is not the main target of this article, this work is also relevant with regards to the computational modeling of language acquisition. The general question of how certain sub-symbolic communication signals can be associated to their meanings through interaction has been largely studied in the literature so far. But the specific question of how teaching signals (speech words here) can be mapped to teaching meanings and how they can be used for learning new tasks has, to our knowledge, not been modeled computationally so far. 

As argued in the previous section, the work we present has also some relevance with regards to the computational modeling of language acquisition. The literature on computational modeling of language acquisition by machines and robots is large and diverse, and focused on many aspects of language learning \cite{steels2012grounding,steels2002aibos, cangelosi2010integration, kaplan2008computational, steels2003evolving, brent1997computational, yu2007unified}. An important line of work investigated the Gavagai problem \cite{quine1964word}, i.e. the problem of how to guess the meaning of a new word when many hypothesis can be formed (out of a pointing gesture for example) and it is not possible to read the mind of the language teacher. Various approaches were used, such as constructivist and discriminative approaches based on social alignment \cite{steels06spatialLanguage, steels2008can}, pure statistical approaches through cross-situational learning \cite{xu2007word, smith2008infants} or more constrained statistical approaches \cite{roy2005semiotic, yu2007unified}. In all these existing models, meanings were expressed in terms of perceptual categories (e.g. in terms of shape, color, position, etc) \cite{steels06spatialLanguage, steels2008can,yu2007unified}, or in terms of motor actions \cite{steels2008robot, Massera2010,sugita05a}. This applies to models implemented in robots, such as in \cite{heckmann2009teaching}, where  the robot ASIMO is taught to associate new spoken signals to visual object properties, both in noisy conditions and without the need for bootstrapping. 

Only very few models so far have explored how other categories of word meanings could be learned. \cite{cederborg2011imitating} presented a model where word meaning expressed the cognitive operation of attentional focus, and some models of grammar acquisition dealt with the acquisition of grammatical markers which meaning operates on the disambiguation of other words in a sentence \cite{steels2012fluid}. 
The work we present in this article shows mechanisms allowing a learner to acquire word meaning associated to teaching and guidance in the context of social interaction. Furthermore, while in most models of lexicon  acquisition no tasks are learned (only word-meaning associations are learned), we show mechanisms allowing the learner to leverage learned word meanings to learn novel tasks from a human.

Finally, the work of Steels and colleagues \cite{steels2012grounding,steels2002aibos} have extensively shown the importance language games, instantiating various families of pre-programmed interaction protocols specifically designed to allow robots to learn speech sounds\cite{de2000self,oudeyer2006self}, lexicons \cite{steels2002aibos} or grammatical structures \cite{steels06spatialLanguage, steels2008can}. Other works used similar interactional frames to allow a structured interaction between humans and robots so that new elements of language could be identified and learnt by the robot learner \cite{roy02a,lyon2012interactive,cangelosi06b,yu2004multimodal,cangelosi2010integration,sugita05a,dominey2005learning,cederborg2011imitating}. In particular, it was shown that these interaction protocols fostered efficient language learning by implementing joint attention and joint intentional understanding between the robot and the human \cite{kaplan2006challenges,yu2005role,yu2007unified}, for example leveraging the synchronies and contingencies between the speech and the action flow \cite{rohlfing2006can,schillingmann2011acoustic}. Yet, to our knowledge, in all these existing works the interaction protocols have always been pre-programmed (that is the robot knows how to use and understand them innately, e.g. he knows how the teacher expresses ``good'' or ``bad'' feedback), as well as used by the robot learner to strictly identify new (form,meaning) pairs in the sensorimotor flow. The methods we present here form a basis on which such flexible and adaptive teaching protocols could be learnt.


Active Learning for Teaching a Robot Grounded Relational Symbols
Johannes Kulick and Marc Toussaint and
Tobias Lang
and
Manuel Lopes


%%%%%%%%%%%%%%%%%%%%%%%%%%%%%%%%%%%%%%%%%%%%%%
%%%%%%%%%%%%%%%%%%%%%%%%%%%%%%%%%%%%%%%%%%%%%%
%%%%%%%%%%%%%%%%%%%%%%%%%%%%%%%%%%%%%%%%%%%%%%
%%%%%%%%%%%%%%%%%%%%%%%%%%%%%%%%%%%%%%%%%%%%%%
%%%%%%%%%%%%%%%%%%%%%%%%%%%%%%%%%%%%%%%%%%%%%%
\section{Unsupervised learning}


\subsection{Multimodal learning}


%%%%%%%%%%%%%%%%%%%%%%%%%%%%%%%%%%%%%%%%%%%%%%
%%%%%%%%%%%%%%%%%%%%%%%%%%%%%%%%%%%%%%%%%%%%%%
%%%%%%%%%%%%%%%%%%%%%%%%%%%%%%%%%%%%%%%%%%%%%%
%%%%%%%%%%%%%%%%%%%%%%%%%%%%%%%%%%%%%%%%%%%%%%
%%%%%%%%%%%%%%%%%%%%%%%%%%%%%%%%%%%%%%%%%%%%%%
\section{Ad hoc team}


%%%%%%%%%%%%%%%%%%%%%%%%%%%%%%%%%%%%%%%%%%%%%%
%%%%%%%%%%%%%%%%%%%%%%%%%%%%%%%%%%%%%%%%%%%%%%
%%%%%%%%%%%%%%%%%%%%%%%%%%%%%%%%%%%%%%%%%%%%%%
%%%%%%%%%%%%%%%%%%%%%%%%%%%%%%%%%%%%%%%%%%%%%%
%%%%%%%%%%%%%%%%%%%%%%%%%%%%%%%%%%%%%%%%%%%%%%
\section{The human in the loop}
\label{chapter:related:humanintheloop}


As mentioned before, an important challenge for such interactive systems is to deal with non-expert humans whose teaching styles can vary considerably. Users may have various expectations and preferences when interacting with a robot and predefined protocols or instructions may bother the user and dramatically decrease the performance of the learning system \cite{rouanet2013impact}. 

Although learning from completely generic feedback types is very hard, we follow the approach from \cite{griffiths2012bottom}, which explores a human to human interaction in a categorization task where instruction can only be provided via six unlabeled symbols (thus the meaning of teaching signals are unknown to the learner). This study shows that tutors seem to spontaneously use three main types of instruction in order to help the learner: positive feedback, negative feedback, and concrete instructions (e.g. name of next optimal action). Yet, to our knowledge, nobody has investigated this problem with artificial agent in the context of human-robot interaction.


Yet, this capability is crucial in infant social development and learning, as well as in adult mutual adaptation of social cues. This has been the subject of experiments in experimental semiotics \cite{galantucci2009experimental}, such as in the work of Griffiths et al. \cite{griffiths2012bottom} who conducted an experiment with human learners learning the meaning of unknown symbolic teaching signals. The experimental setup we present here in the context of human-robot interaction is a variant of theirs, where teaching signals are sub-symbolic and not from a pre-determined set.


\section{Learning from unlabelled instruction}



Interestingly, non-invasive brain-computer interfaces (BCIs) have also looked at this problem. EEG-based BCIs have been used successfully to control different devices, such as robotic arms and simulated agents, using self-generated (e.g. motor imagery) and event-related potentials signals (see \cite{millan10} for a review). 
%
Error-related potential is one event-related potential that appears when the user's expectation diverges from the actual outcome. They have been recently used as reward to teach devices a policy that solves a user's intended task \cite{chavarriaga2010learning,iturrate2010robot}.

As in most BCI applications, it is necessary to perform a calibration phase to learn a decoder (e.g. a classifier) that detects this error-related potential in a single trial. This calibration is required mainly due to various characteristics of the EEG signals: the non-stationary nature \cite{vidaurre11}; the large intra- and inter-subject variability \cite{Polich1997}, and the variations induced by the task \cite{iturrate2013task}. This phase hinders the deployments of BCI applications out of the lab and calibration free methods have been identified as an important step to apply this technology in real applications \cite{millan10}. 
There are few BCI applications that are able to calibrate themselves during operation.  For long term operation using motor rhythms, it is possible to adapt the decoder online \cite{vidaurre2010towards}. For P300 spellers, Kindermans et al. proposed a method to autocalibrate the P300 detector by exploiting multiple stimulations and prior information \cite{Kindermans2012a,Kindermans2012b}. However, Kindermans et al, due to the specific interface used (P300Speller) can guarantee that 1 signal out of 6 has a specific meaning. In our work, and in a majority of sequential problems, it is impossible to define a sequence of actions that guarantee a specific ratio of meanings in the received signals.

Here, we allow the robot to learn the meaning of unknown signals without the need of bootstrapping the system with known signals and by considering real natural speech waves data instead of symbolic labels, as well as a human-robot interaction scenario with a real robot. We extend the work presented in \cite{grizou2013robot} by defining an uncertainty measure allowing the agent to plan its actions efficiently which reduces the learning time.
