The weather in space can have substantial effects on everything within the Sun's influence, including Earth. Most of the effects of space weather are mitigated by Earth's atmosphere and magnetic field, which form a barrier that much of the plasma in space cannot penetrate. Spacecraft and astronauts, however, risk exposure to potentially harmful radiation. Space weather can directly affect Earth's atmosphere, such as near Earth's poles where the magnetic field is shaped in such a way that the solar wind can interact with the upper atmosphere and create the aurora. Space weather can be dangerous, as storms from the Sun damage spacecraft and cause power outages on Earth. To protect astronauts and spacecraft from harm, an understanding of the fundamental physics behind space weather is vital. There are many unknown pieces to the puzzle, but through research and experimentation the picture is slowly being pieced together. As mankind looks toward further exploration of the Moon and the planets, the ability to forecast and protect humans from the effects of space weather become even more important.

This research examines new techniques and tools that can be used to study crucial pieces of the space weather puzzle. The emphasis is on ions found in the space environment: the solar wind (keV), pickup ions (10--100 keV), and energetic particles (100 keV--GeV). Each of these particle populations is created by a specific set of processes, described briefly in Chapter~\ref{chap:Particles}, and all have trajectories through space that are closely related to the magnetic field of the Sun and the heliosphere. Within the Alfv\'{e}n radius (10--20 R$_\sun$), where the kinetic energy of the solar wind is weaker than the energy density of the Sun's magnetic field, the solar wind is guided by the field's shape and corotates with the Sun. Beyond this radial distance, the electrically charged and highly conductive solar wind controls and shapes the magnetic field, drawing it radially outward into the heliosphere. Energetic particles travel along the lines of magnetic flux that extend from the solar surface to the outermost reaches of the solar system. To a large degree, if the shape and path of the Sun's magnetic field can be accurately mapped, the trajectories of these particles could be inferred. However, such mapping techniques have been elusive, or inaccessible to the broad community.
