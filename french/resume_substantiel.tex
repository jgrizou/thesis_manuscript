%!TEX root = ../thesis.tex
\define{\imgpath}{french/img}

\chapter*{Résumé substantiel}
\label{chapter:frecnhresume}
\minitoc

Cette thèse s'intérresse à un problème logique dont les enjeux théorique et pratique sont multiples. Ce problème, dans sa forme simple, peut-être présenté ainsi: Imaginez que vous êtes dans un labyrinthe, dont vous connaissez toutes les routes menant à chacune des portes de sortie. Derrière l'une de ces portes se trouve un trésor; mais vous n'avez le droit d'ouvrir qu'une seule porte. Un vieil homme habitant le labyrinthe connait la bonne sortie et se propose alors de vous aider l'identifier. Pour cela, il vous indiquera la direction à prendre à chaque intersection. Malheureusement, cet homme ne parle pas votre langue, les mots qu'il utilise pour dire ``droite'' ou ``gauche'' vous sont donc inconnu. Est-il possible de trouver le trésors et de comprendre l'association entre les mots du vieil homme et leurs significations ?

Ce problème, bien qu'en apparence abstrait, est relié à des problématiques concrètes dans le domaine de l'interaction homme-machine et que nous présentons aux chapitres~\ref{chapter:introduction} et \ref{chapter:relatedwork}. En effet, si nous renversons les rôles: un human, prenant la place du vieil homme, souhaite guider un robot vers la bonne sortie du labrytinthe. Ce robot ne sait donc pas en avance quel est la bonne sortie mais il sait oû se trouvent toutes les portes et comment s'y rendre. Imaginons maintenant que ce robot ne comprenne pas a priori le language de l'humain; en effet il est très difficile de construire un robot à même de comprendre parfaitement chaque langue, accent, et préférence de tout un chacun. Il faudra alors que le robot apprenne l'association entre les mots de l'humain et leur sens, tout en réalisant la tâche que l'utlisateur humain lui indique (e.g. trouver la bonne porte). Ce problème n'est pas simple car pour comprendre le sens des signaux il faudrait connaitre la tâche, et pour connaître la tâche il faudrait connaitre le sens des signaux.

Il s'agit donc, pour un labyrinthe donné, de trouver la suite d'action permettant de collecter suffisament d'information de la part de l'humain pour comprendre à la fois le sens de ses mots et la porte derrière laquelle se cache le trésor. Cela dépend donc de la configuration du labyrinthe et de l'historique complet de l'interaction entre les deux protagonistes.

Dans cette thèse nous présentons une solution à ce problème. Pour cela nous faisons dabord l'hypothèse qu'un nombre fini de tâche est défini et connu de l'homme et de la machine, i.e. un nombre fini de portes existe. Nous supposons également que le robot dipose d'un modèle de la logique de l'utilisateur et est donc capable de faire le raisonement suivant: Si l'human veut que j'aille vers la Porte $A$ alors lorsque je suis à l'intersection $I$, il devrait logiquement me dire d'aller dans la direction $D$. Noter que cette phrase commence par une supposition sur la tâche, qui n'est en aucun cas connu à l'avance. Ainsi, le robot étant équipé de plusieurs hypothèses (Porte $A$, $B$, $C$, ..), lorqu'il se trouve à l'intersection $I$, l'utilisateur prononce un mot (e.g. "wadibou"), dont autant d'interpretations sont faite que d'hypothèses sur la tâche.

Notre hypothèse sous-jacente est que l'utilisateur humain est logique et cohérent tout au long de l'interaction, utilisant ainsi toujours le même mot pour dire la même chose. Il nous faut donc tenir compte de tout l'historique de l'interaction pour analyser quels mots auraient été utilisé pour dire quoi selon chaque hypothèse de tâche. Nous comprenons ainsi que, sous certaine conditions qui sont explicitées au chapitre \ref{chapter:lfui}, il est possible d'éliminer toutes les hypothèses générant des interprétation du sens des signaux incohérentes. L'unique hypothèse restante nous informera donc à la fois de la bonne tâche, i.e. la bonne porte à ouvrir, mais aussi de la bonne association entre les mots de l'utilisateur et les sens qui y sont associés, i.e. du language de l'utilisateur.

Etendons maintenant ce problème à d'autres modalité d'interaction et d'autre domaine d'applications.

facilemet transposable a dautre signaux, parole: pas d'apriori sur le sens des mots

Dans cette thèse nous étudions comment il est possible de créer des interfaces ne nécessitant pas de phase de calibration.

Remplaçont maintenant l'example du labyrinthe par une tâche plus concrête, plus utile. Une personne au capacités de communication réduite doit utiliser une machine pour communiquer avec le monde exterieur, il doit donc pouvoir la commander.

Prenons example d'une personnes fortement handicapée ne pouvant communiquer avec le monde exterieur que par de fragile clignement des yeux ou en aillant recours à l'enregistremnt de leur ondes cérébrale. Il devient alors difficile, voir même impossible de savoir à l'avance les intentions de communications de ces personnes. Il est donc primordiale de disposer de machine qui sont à même de s'apater automatiquement à chuaque personne.

Et il est interressant de voir que c'est la communaute BCI qui s'est interresse le plus à ce problème. En effet, a l'opposé des mode d'interaction classique, telle que la parole, les gestes, ou les expression faciale, nous n'avons aucun apriori sur l'utilisation des signaux du cerveau. Le problème de l'auto-calibration est alors plus flagrant dans le domaine des BCI.

Etrangement, cette thèse ne traite pas directement du problème simple, symbolique, mais s'interresse directement par une représentation non-symbolique des signaux de communications. Ceci dans n but applicatif à court termes auquel de tedieuse preuves mathématiques dans des domaine trop simplifé n'aurait laissé guère de temps à l'experimentation. Ainsi la formulation simple du labyrinthe présenté en début de ce résumé n'est adréssé que dans la toute dernière section de cette thèse ainsi qu'une preuve de la validité de notre solution pour le cas de signaux de communication symbolique.

le planning est égelemtn différent des algortihmes précedemment dévellopé. En effet une couche d'incertitude supplémentaire est présente, le sens des signaux est inconnue et nous disposont d'une multitude de candidat possible. Il faut donc inclure cette incertitude dans la mesure d'incertidtude globale permettant de naviguer plus efficacement dans le monde. afin de collecter des signaux varié.

\subsection*{Résultats}

Nous appliquons nos algortihmes d'auto-calibration à deux examples protypique de l'interaciton homme-robot et de l'interactin cerveau-machine.

Nottre aprochce est tres generique et permet de commencer a interagir avec une machine pour la résolution du tache sequentiel sans cque le systeme comprenne par avance les signaux de communiatins e l'utilisateur. Ces signaux peuvent etre de differente type, vocaux, gestes, onde cerebrale et doivent pouvoir être représenté sous la forme d'un vecteur.

Au chapitre bci, nous présentons l'application principale de ce travail au interface cerveau-machine. Ce genre d'interface permet au personne a fort handicap d'interagir avec le monde exterieur par le bias de son cerveau. Plus precisement, nous pouvons enregistrer des varitions de potentiel à la surface du cerveau. Ces ondes ont des porpritété différent en fonction de l'activité mentale du sujet et il est possible de différentier des activité motrice et même des signaux d'erreur de type oui/non. Le problème de ces systeme est qu'ils ne sont pas universelle et doivent être adapté à chaque utilisateur. Ceete adaptation est faite par une periode de calibration ou l'utilisateur, souvent ennuyeuse, et durent laquelle le systeme est inutilisable et neccessitant l'intervention d'une personne exterieur. De plus, cette pahse de calibration doit être effectué régulièrmeent car les signaux varie de jour en jour. Mais aussi pas example lapossition du casqua EEG, necessitant une calibration quotidienne de ce genre d'interfcae.

Nous avons appliqué nos algortihm d'auto-calibration au interface cerveau utilisateur, testé en simulation mais aussi avec de vrai patient. resultat avec des vrai sujets en temps réel

Nos résultats montre que notre approche est fonctionnelle et permet une utilisation pratique  de l'interface plus rapidement. De plus notre systemem ne necessite pas la presence d'une perosnne exterieur pour la pahse de calibraiton et est donc un candidat potentiel pour ammener l'utilisation des interface cerveau mahcine dans les maisons.

Enfin au chapitre \ref{chapter:bci}, nous présentons les résultats d'experiences réel dans le cadre de l'interaction cerveau-machine. Ce sont donc des sujets humanin qui ont pour tâche de guider un agent dans un labyrinthe en lui indiquant si ces actions sont ``correct''  ou ``incorrecte'' vis a vis de l'objectif défini simplement en pensant à ``correct'' ou ``incorrect'' dans leur esprit. Les ``pensées'' de l'utilisateur sont mesuré par le biais d'électrode au contact de son cerveau. Gràce a un procédé décris au chapitre \ref{chapter:bci}, il est possible de représenter ces signaux afin différentier ceux signifiant correct de ceux signifiant incorrect. L'agent ne connait donc ni la tâche à effectuer ni le mapping entre les ondes cérébrales et leur sens (``correct'' ou ``incorrect'').

\subsubsection*{Extensions}

Au chapitre~\ref{chapter:limitations}, nous abordons et porposons des solutions à de multiples limitations de l'approche présentée dans cette thèse. Nous montrons d'abord qu'il est possible d'utiliser le systeme dans des espaces continue, premièrement pour un état continue du système mais aussi sur un ensemble infini d'hypothèse sur la tâche. Par la suite, nous montrons que la connaissance a priori du protocole d'interaction  n'est pas une limitation forte et que notre systeme peut ainsi detecter le bon protocole par l'interaction pratique avec l'utlisateur. Finalement, nous démontrons mathematiquement sur un example simplifié et sous certaine conditions que notre méthode est garanti d'identifier la bonne hypothèse. Ce dernier dévelopement montre que ce genre de problème peut-être modélisé mathématiquement et ouvrant la voie de prochaine exploration plus théorique de ce problème permettant peut-être de trouver des  guarantie plus grande sur la convergence et performance de nos algorthmes, Bien que cela semble encore très limité de par la nature imprevisible du comportement humain.

\subsection*{Expèrience humain-humain}

Une question naturelle est de ce demander si une interaction aussi peu contrainte pourra fonctionner avec deux humains. Nous avons donc mis en place une experience de ce type avec deux humains devant intéragir par le biais d'une interface simple dont le sens des signaux était inconnue au départ pour les deux parties.

Nous appelons l'utilisateur du qui appuie sur les boutons et connais la tàche, l'architecte. L'utlisateur qui observe les signaux et agit sur le monde est appelé le constructeur.

Il est intérressant de voir la construction dun language commun, non prévu au début

Une personne extèrieure à l'experience ne pourra pas comprendre ce qui ce passe en observant le résultat final de l'intéraction, sans connaitre l'historique.

\subsection*{Conclusion}

La vision dévellopé dans cette thèse est qu'il est possible pour une machine d'executé les souhait d'un utilisateur sans comprende la facon précise dont l'utisateur communique l'information. Plus concretement notre systeme n'a pas d'apriori sur le sens des signaux reçu et construit son modèle en interaction avec l'utilisateur sans jamais avoir access a une source sure d'information.

Au dela du challenge technique de l'auto-calibration, des questions d'utilisation pratique et d'acceptabilité eclose qui sont preesenté au chapitre~\ref{chapter:conclusion}.

Comment les gens vont réagir au fait que la machine, le robot, ne soit plus apte immédiatement mais doivent apprendre le sens des signaux. Cela sera, nous l'espèrons, le fruit de nombreux travaux futur.\\

\noindent {\large\textbf{Mots-clés:}} Auto-Calibration, Apprentissage par interaction, Intéraction Human-Robot, Interface Cerveau-Machine, Intéraction Intuitive et Adaptative, Robotique, Acquisition de Symbole, Apprentissage Actif, Calibration.\\

Ce travail a été financé par INRIA, le Conseil R\'egional d'Aquitaine et la bourse ERC EXPLORERS 24007.
