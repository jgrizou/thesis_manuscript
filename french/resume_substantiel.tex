%!TEX root = ../thesis.tex

\chapter*{Résumé substantiel}
\label{chapter:frecnhresume}
\minitoc

suivre l'organisation de l'abstract.

\subsection*{Introduction}

Problème de l'intéraction humain-machine.

Pour interagir il faut que la machine comprenne le sens des signaux de communication de l'humain.

Pour apprendre le sens des signaux,  il faut savoir ce que l'humain veut dire pour construire un modèle statistique des signaux.

Dans cette thèse nous étudions comment il est possible de créer des interfaces ne nécessitant pas de phase de calibration.


\subsection*{Principe algorithmique}

Pour cela, nous allons mettre en place un système d'hypothèse sur la tâche que veut effectuer l'utilisateur. Et créer autant d'interpretation que de tâche possible. Notre hypothèse subjacente est que seule l'hypothèse correcte, i.e. celle suivi par l'utilisateur vera une cohérence entre les labels assignés aux données et la structure de ces données dans l'espace de features.

le planning est égelemtn différent des algortihmes précedemment dévellopé. En effet une couche d'incertitude supplémentaire est présente, le sens des signaux est inconnue et nous disposont d'une multitude de candidat possible. Il faut donc inclure cette incertitude dans la mesure d'incertidtude globale permettant de naviguer plus efficacement dans le monde. afin de collecter des signaux varié. 

\subsection*{Résultats}

BCI

time line

resultat avec des vrai sujet en temps réel

\subsubsection*{Extensions}

facilemet transposable a dautre signaux, parole: pas d'apriori sur le sens des mots

espace d'état continue

espace des taches continue

hypothèse sur le protocole d'interaction

preuve

\subsection*{Expèrience humain-humain}

Une question naturelle est de ce demander si une interaction aussi peu contrainte pourra fonctionner avec deux humains. Nous avons donc mis en place une experience de ce type avec deux humains devant intéragir par le biais d'une interface simple dont le sens des signaux était inconnue au départ pour les deux parties. 

Nous appelons l'utilisateur du qui appuie sur les boutons et connais la tàche, l'architecte. L'utlisateur qui observe les signaux et agit sur le monde est appelé le constructeur.

Il est intérressant de voir la construction dun language commun, non prévu au début

Une personne extèrieure à l'experience ne pourra pas comprendre ce qui ce passe en observant le résultat final de l'intéraction, sans connaitre l'historique.

\subsection*{Conclusion}

Cela fonctionne, mais est-ce vraiment utilisable en pratique

Il manque des experiences utilisateur avancées.
Comment les gens vont réagir au fait que la machine, le robot, ne soit plus apte immédiatement mais doivent apprendre le sens des signaux.
